\chapter*{Závěr}
\phantomsection
\addcontentsline{toc}{chapter}{Závěr}

V teoretickém úvodu byly rozebrány jednotlivé technologie senzorů pro měření daných veličin a~možnosti přenášení naměřených dat na server. Následně byly vybrány všechny potřebné senzory s~ohledem na kvalitativní parametry a~co nejnižší provozní spotřebu pro dosažení co nejdelšího provozu při případném provozu z~akumulátorů. Jako IoT síť pro přenos dat byla pro první experimentální pokusy vybrána síť WiFi, s~možností doplnění připojení přes LoRaWAN. Nedílnou součástí výběru hardwarových kompnent byl výběr řídícího mikrokontroleru, kde bylo zapotřebí vybrat vhodný typ podle potřebných komunikačních sběrnic a~hardwarových prostředků. Nakonec byl vybrán mikrokontroler ESP32, jelikož má všechny potřebné periferie a~umožňuje dosáhnout při různých provozních režimech velice nízké spotřeby.

Po sestavení výsledného blokového schématu byla navržena dvouvrstvá deska plošných spojů o~rozměrech $100\times\SI{100}{\milli\metre}$, která byla následně vyrobena včetně strojního osazení většiny součástek. Po jejím vyrobení byly doosazeny ručně zbylé součástky a~byla vyzkoušena alespoň základní funkčnost hlavních komponent. Bylo zjištěno, že mikrokontroler je možné naprogramovat a~funguje komunikace se senzory. Byla změřena také spotřeba celého zařízení při vypnutí všech senzorů a~ponechání pouze zapnutého mikrokontroleru v~režimu hlubokého spánku. V~tomto stavu má deska spotřebu okolo \SI{500}{\micro\ampere}. Dále byla změřena impedance integrované antény na desce plošných spojů, která ovšem nevyšla podle očekávání. Změřená impedance byla $700+j\SI{0}{\ohm}$, což není správné. Může to být nejspíše způsobeno špatným návrhem desky plošných spojů.

Dále je tedy třeba naprogramovat firmware mikrokontroleru, aby bylo možné měřit a~odesílat všechny potřebné veličiny. Je třeba vyřešit problém s~integrovanou anténou, a~to například vyrobením desek, na kterých bude pouze zkoumaná anténa. Eliminuje se tím vliv okolí na desce a~může se poté provést nové měření. V~neposlední řadě je také potřeba do návrhu přidat možnost napájet zařízení z~baterie pro použití mimo dosah elektrické sítě a~s~tím je spojená další optimalizace spotřeby celého zařízení.