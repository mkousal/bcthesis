\chapter*{Závěr}
\phantomsection
\addcontentsline{toc}{chapter}{Závěr}

V teoretickém úvodu byly rozebrány jednotlivé technologie senzorů pro měření daných veličin a~možnosti přenášení naměřených dat na server. Následně byly vybrány všechny potřebné senzory s~ohledem na kvalitativní parametry a~co nejnižší provozní spotřebu pro dosažení co nejdelšího provozu při provozu z~baterie. Jako síť pro přenos dat byly vybrány dvě možnosti. První je LoRaWAN ve spojení s komunitním serverem The Things Network, tato síť umožňuje připojení zařízení a odesílání dat kdekoliv, kde se vyskytuje gateway zapojená do tohoto projektu. Druhou sítí, použitelnou například v domácích podmínkách, je WiFi. Nedílnou součástí výběru hardwarových komponent byl výběr řídícího mikrokontroleru, kde bylo zapotřebí vybrat vhodný typ podle potřebných komunikačních sběrnic a~hardwarových prostředků. Nakonec byl vybrán mikrokontroler ESP32, jelikož má všechny potřebné periferie a~umožňuje dosáhnout při různých provozních režimech velice nízké spotřeby.

Po sestavení výsledného blokového schématu byly navrženy obvodové zapojení a také desky plošných spojů v programu KiCad. Kromě hlavní řídící desky, která má rozměry $50 \times 50$\SI{}{\milli\metre} a obsahuje většinu senzorů a veškerou řídící elektroniku. Tuto hlavní desku doplňuje menší deska s dvěma senzory pro měření parametrů osvětlení a je nasazena na tuto hlavní desku. Celé zařízení bylo navrhováno s ohledem na co nejnižší spotřebu a tudíž i provoz na baterie. Z tohoto důvodu jsou vypínatelné veškeré senzory a komunikační moduly tak, aby v režimu hlubokého spánku byl v provozu pouze časovač v mikrokontroleru, který zařízení po definovaném čase opět probudí. Díky tomuto návrhu se podařilo vyrobit zařízení, které má v režimu hlubokého spánku spotřebu pouhých \SI{5.8}{\micro\ampere}. Při napájení baterií LiFePO4 s kapacitou \SI{3200}{\milli\ampere\hour} a periodou měření a odesílání dat \SI{1}{\hour} je teoreticky možné dosáhnout výdrže až 160~dní.

Pro možnost nabíjení baterie přímo v zařízení bez nutnosti jej rozebírat byla dále navržena nabíjecí deska. Tato nabíječka umožňuje připojení k hlavní desce i v průběhu její činnosti. Nabíječku lze napájet napětím \SI{5}{} až \SI{40}{\volt}. Na její vstup může být připojen solární panel a aktivována integrovaná MPPT funkce obvodu. S tímto zapojením je tak možné i s velice malým solárním panelem dosáhnout nepřetržité funkce zařízení.

Po sestavení a oživení zařízení bylo zapotřebí naprogramovat potřebné funkce. Firmware je založen na frameworku ESP-IDF od samotného výrobce čipu Espressif. Byly napsány veškeré obslužné knihovny pro senzory a zajištěna možnost uživatelského výběru a konfigurace připojení k síti LoRaWAN či WiFi. Při psaní firmwaru bylo také dbáno na optimalizaci spotřeby zařízení a zejména co nejkratší běh samotného měření a odesílání. Proto jsou využity vícevláknové operace a celý cyklus od probuzení mikrokontroleru přes měření až po odeslání dat trvá v průměru \SI{30}{\second}.

Důležitým rozhodnutím pro celou práci bylo, jakým způsobem uživateli předat v grafické podobě naměřená data. V práci byly rozebrány různé možnosti veřejně dostupných serverů. Nakonec bylo vybráno vlastní řešení poskládané z konkrétních služeb a aplikací, které jsem nainstaloval na Raspberry Pi 4. Toto vlastní řešení má výhodu v teoreticky neomezeném množství přenesených dat a připojených zařízení. Pokud by přestalo toto Raspberry Pi výkonostně stačit, stačí tyto služby pouze nainstalovat například na pronajatý server a lze celé řešení výkonnostně škálovat. Nevýhodou tohoto Rapsberry Pi je, že pokud vypadne elektřina nebo internet, tak se posílaná data ztratí a nebudou přijaty. Tento neduh může opět odstranit hostování v cloudu.

Poslední prací bylo navrhnutí a vytištění krabičky na 3D tiskárně pro veškerou elektroniku a senzory. Poté bylo sestaveno celé zařízení a zprovozněno. Celkové náklady na sestavení se v květnu 2022 pohybovaly okolo $1630$~Kč. V ceně není zahrnuta nabíječka ani 3D tisk krabičky.