\chapter{Teoretická část}

V teoretické části práce se zaměřím na teoretické principy, na kterých pracují senzory pro monitoring ovzduší. Převážná část z použitých senzorů využívá nepřímý způsob měření námi požadovaných veličin, kde se většinou změna dané veličiny projeví na senzoru změnou odporu a tím i napětí či proudu jím protékajícím.

\section{Měření oxidu uhelnatého}

Oxid uhelnatý je jedovatý plyn bez chuti a zápachu. Vzniká nejčastěji při nedokonalém hoření převážně pevných paliv ale i plynů, proto je třeba jeho hodnotu hlídat. Udává se, že zhruba od 100 ppm je u většiny lidí přítomen nějaký ze symptomů otravy tímto plynem (bolest hlavy, únava, nevolnost).

Oxid uhelnatý lze měřit více způsoby. Nejpřesnější možností je optický senzor využívající infračervené světlo. Tento typ senzoru je založen na základě měření rozdílu intenzity infračerveného záření o dané vlnové délce. Přiváděný plyn je osvětlován infračerveným zářením, které je přítomnými molekulami oxidu pohlcováno a poté je přes reflexní vrstvu odraženo zpět do snímače, kde je umístěn pyrodetektor, který převádí intenzitu tohoto světla na elektrický signál. Se vzrůstající koncentrací klesá intenzita světla dopadajícího na povrch pyrodetektoru. Tento princip měření je nejpřesnější, podává stabilní výsledky a má dlouhou životnost. Bohužel je velice drahý a tak jej není možné použít v domácích zařízeních.

Další z možností měření je elektrochemický senzor. Takovýto senzor pracuje na principu měření proudu vznikajícího reakcí sledovaného plynu s elektrolytem, který je obsažen uvnitř senzoru. Při konstrukci takovéhoto senzoru je třeba zvolit elektrody a elektrolyt tak, aby na jedné z elektrod docházelo k chemické reakci, která vyvolá změnu proudu. Tato změna je poté následně zesílena do měřitelné podoby a odpovídá koncentraci oxidu uhelnatého. Bohužel díky nutnosti chemické reakce několika přítomných látek není tento druh senzoru možné zkonstruovat pro dlouhou životnost. Spolu s tímto neduhem je zde také časová nestálost podávaných výsledků kvůli ubývání elektrolytu a opotřebení měřících elektrod. Životnost takového senzoru je tedy maximálně v řádu několika málo roků. 

Poslední a zároveň nejlevnější možností měření koncentrace oxidu uhelnatého je použití polovodičových senzorů. Polovodičový přechod u těchto senzorů je vyroben tak, aby se při přítomnosti sledovaného plynu změnila jeho vodivost. Na základě této změny jsme poté schopni změřit napětí a proud na přechodu, čímž můžeme určit koncentraci CO. Nevýhodou těchto systémů je ovšem jejich relativní nepřesnost a hlavně nelineární průběh měřeného signálu. Jsou ovšem díky své ceně snadno použitelné a dostupné v komerčně prodávaných detektorech do domácností a pro laická měření. Pro exaktní měření je ovšem nutná jejich častější kalibrace vůči známé koncentraci měřeného plynu.

\section{Měření koncentrace prachových částic}

Prachové částice je možné rozdělit do několika kategorií podle jejich velikosti. Často se lze setkat s pojmem např. PM2.5, což je zkratka z anglického particulate matter (pevné částice) a číslo, které udává maximální velikost těchto částic v \SI{}{\micro\metre}. Nejčastěji se měří částice do velikosti \SI{10}{\micro\metre}, \SI{2,5}{\micro\metre} a \SI{1}{\micro\metre}. Částice o velikosti \SI{10}{\micro\metre} nejsou pro lidský organismus příliš škodlivé, lidské tělo jich většinu dokáže zachytit již při vstupu do dýchacích cest. Problém nastává při vyšší koncentraci částic o velikosti \SI{2,5}{\micro\metre}. Zde již tělo nemá přirozenou obranu a dostávají se tak přímo do plic. Částice menší než \SI{0,5}{\micro\metre} jsou schopny proniknout až do krevního řečiště.

Nejčastěji se v praxi měří koncentrace částic o velikosti \SI{10}{\micro\metre} a \SI{2,5}{\micro\metre}. Všechna zařízení pro měření těchto částic fungují na principu pohlcování či odrážení světelného paprsku. Pro měření je tedy potřeba zdroj světla a detektor světelného paprsku. Jako zdroj se používají LED nebo stále častěji laser. Princip měření tedy spočívá v osvícení vzorku vzduchu daným paprskem světla, který se o prachové částice ve vzorku rozptýlí nebo pohltí. Množství dopadeného světla je tedy nepřímo úměrné koncentraci prachových částic v daném vzorku. V principu jsme schopni měřit tak malé částice, jak přesný zdroj světla (šířka paprsku) jsme schopni vyrobit a také jej potom detekovat.

Dříve používané LED mají nevýhodu v tom, že vyzařují široký paprsek světla, který nejsme schopni jednoduše soustředit do jednoho bodu. Lze využít optickou soustavu pro zaostření takového paprsku světla, ovšem v daném prašném prostředí by docházelo k častému opotřebení a zaprášení čoček, které by poté ztrácely své vlastnosti a měření by bylo nemožné. Z těchto důvodů je v dnešní době více používanější laser, jelikož jsme schopni vytvořit paprsek o dané vlnové délce, výkonové hustotě a velikosti. 

Poslední součástí těchto detektorů je mechanismus, kterým se do senzoru dostává čerstvý vzorek vzduchu. Nejjednodušší je využití malého ventilátoru, který bude do prostoru senzoru vhánět čerstvý vzduch z okolí. Nevýhodou takového řešení je hlučnost senzoru a také možnost zanášení senzoru nečistotami z okolí. Proto se objevují i senzory, které mají tento ventilátor nahrazeny topným elementem (nejčastěji výkonový rezistor), kterým protéká proud a ohřívá vzduch okolo. Ten pak díky rozdílné hustotě teplého a studeného vzduchu začne stoupat vzhůru a unáší s sebou prachové částice do měřeného prostoru. Zde je ovšem třeba dávat pozor na konstrukci takového senzoru a na výrobcem předepsané požadavky na montáž, jelikož jej nelze umístit téměř libovolně v prostoru, jako tomu může být u senzoru s ventilátorem.

\section{Intenzita osvětlení}
