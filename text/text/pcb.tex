\chapter{Návrh desek plošných spojů}

Návrh PCB (desek plošných spojů) byl proveden podle výrobních pravidel firmy JLCPCB\footnote{\url{https://jlcpcb.com}}. Maximální rozměry pro levnou výrobu (2 až 4\$ za 5~ks desek) jsou $\SI{100}{}\times \SI{100}{\milli\metre}$ pro dvouvrstvou desku. Do těchto výrobních možností je tedy nutné koncipovat veškeré návrhy desek. Desky budou navrženy pro ruční osazení a jsou tomu tedy i uzpůsobeny pouzdra součástek. Tuto cestu jsem zvolil, protože i po započítání nákladů na dopravu a clo je výroba u této společnosti nesrovnatelně cenově výhodnější, než u české konkurence. Veškeré návrhy PCB a schémat jsou provedeny v programu KiCad\footnote{\url{https://kicad.org/}}, který je dostupný zdarma a~lze jej používat i~pro komerční projekty. Navíc je open-source a má veřejně dostupné Python API (Application Programming Interface), což umožňuje komunitě vytvářet pluginy a rozšíření pro usnadnění práce.

\section{Návrh desky plošných spojů nabíječky}

Po návrhu schematického zapojení samotného nabíjecího obvodu bylo zapotřebí navrhnout desku plošných spojů. Jelikož se jedná o jednoduchý obvod s málo součástkami, vystačí dvouvrstvá deska plošných spojů. Na desce jsou navrženy celkem tři druhy možností napájení. Je zde konektor Micro-USB, USB-C a také dvě plošky pro připojení jakéhokoli zdroje \SI{5}{} až \SI{40}{\volt}. Poslední ze zmíněných možností se hodí právě pro připojení jakéhokoliv solárního panelu, který splňuje bude mít dostatečný výkon. Společně s možností solárního panelu je zde také umístěna pájecí propojka, kterou lze zapnout nebo vypnout MPPT funkci obvodu. Pokud uživatel chce funkci vypnout (napájení ze stabilního zdroje napětí), stačí pájením spojit prostřední plošku a plošku s nápisem OFF. Při zapnutí je třeba doplnit dva rezistory do napěťového děliče (R10 a R11 podle schématu \ref{fig_ChargerSchematic}) dle rovnice \ref{eq_MPPTRegulator}. Pro připojení nabíječky k hlavní desce jsou použity \SI{2}{\milli\metre} banánky, které zajistí mechanické zajištění a navíc snesou proudové zatížení až \SI{15}{\ampere}, takže nebudou vznikat ztráty přechodovým odporem. Při návrhu spojů bylo dbáno na optimální rozložení součástek z hlediska proudových smyček dle datasheetu výrobce čipu. Výsledná deska má rozměry $\SI{40}{}\times \SI{40}{\milli\metre}$ a je vidět na obrázku \ref{fig_chargerBoardKiCad}.

\begin{figure}[h]
    \centering
    \subfloat[][]{\includegraphics[width=0.45\textwidth]{obrazky/batteryCharger-top_.png}}
    \subfloat[][]{\includegraphics[width=0.55\textwidth]{obrazky/batteryCharger-final.png}}
    \caption{}
    \label{fig_chargerBoardKiCad}
\end{figure}




% Deska plošných spojů byla navržena podle výrobních pravidel firmy JLCPCB\footnote{https://jlcpcb.com/}. Maximální rozměry jsou $\SI{100}{}\times \SI{100}{\milli\metre}$ a~byla využita dvouvrstvá deska. Do těchto výrobních možností je tedy nutné koncipovat celý návrh desky plošných spojů. Osazování proběhne v~drtivé většině součástek u~dříve zmíněné firmy strojně, jelikož i~za cenu včetně osazení je cena součástek nakoupená právě u~nich nesrovnatelná například s~českou konkurencí. Jednou z~možných nevýhod je absence strojního osazení z~obou stran desky, tento výrobce umožňuje osazovat pouze jednu stranu desky plošného spoje.

% Pro dosažení co nejlepších parametrů celého obvodu bylo dbáno na doporučené zapojení dle datasheetu výrobců. Největší pozornost byla věnována oblasti spínaného zdroje, jelikož zde je důležité správně rozložit součástky a~dimenzovat spoje na desce. Pro dosažení nízké spotřeby je možné jednotlivé skupiny senzorů a~modulů vypínat. Na obrázku \ref{fig_PCB-top} je vidět celá navržená deska plošných spojů o~rozměrech $\SI{100}{}\times \SI{100}{\milli\metre}$ z~horní strany. Výsledná deska plošných spojů byla navržena v~programu KiCad\footnote{https://www.kicad.org/}, který je dostupný zdarma a~lze jej používat i~pro komerční projekty.

% \begin{figure}[h]
%     \centering
%     \includegraphics[width=0.7\textwidth]{obrazky/PCB_top.png}
%     \caption{Pohled na 3D model desky plošného spoje shora.}
%     \label{fig_PCB-top}
% \end{figure}