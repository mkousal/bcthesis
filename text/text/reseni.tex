\chapter{Návrh zařízení}

Tato kapitola se bude zabývat návrhem hardwaru celého zařízení. Jedním z hlavních požadavků je nízká spotřeba, kterou bude třeba zohlednit při vybírání použitých senzorů, řídícího mikroprocesoru, komunikačního modulu a ostatních obvodových komponent. 

\section{Výběr senzorů}
Celé zařízení bude schopno měřit koncentraci prachových částic, koncentraci oxidu uhelnatého, intenzitu osvětlení, intenzitu UV záření, teplotu, atmosférický tlak a relativní vlhkost. V následujících částech tedy bude popsán výběr z dostupných senzorů.

\subsection{Senzor koncentrace prachových částic}

Na trhu je dostupných hned několik senzorů na měření koncentrace prachových částic. Jak již bylo zmíněno v teoretickém úvodu, budu vybírat senzor, který tuto koncentraci určuje na základě osvícení daného vzorku vzduchu a následně měření odraženého světla. Nyní vyvstává otázka, jestli zvolit senzor, který bude vzorek vzduchu vhánět do měřícího prostoru nuceně pomocí ventilátoru nebo jen za využití stoupání teplého vzduchu nahoru. V tabulce \ref{tab_DustSensors} jsou uvedeny senzory, které jsou relativně cenově přijatelné, a~daly by se pro neprofesionální měření využít.

\newcommand{\ugcm}{\micro\gram\per\cubic\meter} % define new command used in '\SI' -> micro gram per cubic meter

\begin{table}[h]
    \centering
    \begin{tabular}{c|cccc}
    \textbf{Název}           & \textbf{Rozlišení}                & \textbf{Přesnost}                              & \textbf{Proud}                           & \textbf{Čas čtení}                \\ \hline
    \multirow{2}{*}{PMS5003} & \multirow{2}{*}{\SI{1}{\ugcm}}    & $0-100$\SI{}{\ugcm}:$\pm$\SI{10}{\micro\gram}  & \multirow{2}{*}{\SI{100}{\milli\ampere}} & \multirow{2}{*}{\SI{10}{\second}} \\
                             &                                   & $100-500$\SI{}{\ugcm}:$\pm$\SI{10}{\percent}   &                                          &                                   \\ \hline
    \multirow{2}{*}{PM1003}  & \multirow{2}{*}{\SI{1}{\ugcm}}    & $0-100$\SI{}{\ugcm}:$\pm$\SI{30}{\micro\gram}  & \multirow{2}{*}{\SI{90}{\milli\ampere}}  & \multirow{2}{*}{\SI{30}{\second}} \\
                             &                                   & $100-500$\SI{}{\ugcm}:$\pm$\SI{30}{\percent}   &                                          &                                   \\ \hline
    \multirow{2}{*}{PM1006}  & \multirow{2}{*}{neuvedeno}        & $0-100$\SI{}{\ugcm}:$\pm$\SI{20}{\micro\gram}  & \multirow{2}{*}{\SI{30}{\milli\ampere}}  & \multirow{2}{*}{\SI{8}{\second}}  \\
                             &                                   & $100-500$\SI{}{\ugcm}:$\pm$\SI{20}{\percent}   &                                          &                                   \\ \hline
    GP2Y1010AU0F             & \SI{0,5}{\volt}$/$\SI{100}{\ugcm} & záleží na ADC                                  & \SI{20}{\milli\ampere}                   & \SI{1}{\second}                           
    \end{tabular}
    \caption{Porovnání vybraných parametrů senzorů koncentrace prachových částic}
    \label{tab_DustSensors}
\end{table}

Po pečlivém prostudování jednotlivých parametrů jsem zvolil senzor PMS5003 od firmy PLANTOWER. Důležitým aspektem při výběru byla také cena tohoto senzoru, v době vypracovávání této práce jej šlo pořídit za zhruba 350~Kč. Dalším důležitým parametrem byla spotřeba proudu v aktivním stavu, na první pohled se může zdát, že oproti všem senzorům má spotřebu nejvyšší. Oproti PM1003 má však třetinový čas potřebný k získání měřených dat, takže spotřebovává sice vyšší proud, ale po kratší časový úsek. PM1006 má spotřebu proudu zhruba třetinovou, ale vzorek měřeného vzduchu se do měřícího prostoru dostává pomocí sálání a tak je nutné zajistit konstrukčně dostatečně a správně dimenzované průduchy a také konstantní polohu a hlavně náklon senzoru, což by mohlo být v praxi téměř nemožné, pokud má být zařízení používáno také ve volném prostranství. Poslední ze senzorů v tabulce \ref{tab_DustSensors} má zdánlivě nejlepší výsledky. Bohužel se jedná pouze o měřící modul samotný, který neobsahuje žádný ventilátor ani řídící logiku, je tedy třeba tyto věci zapojit a konstrukčně vyřešit. Nejjednodušší na obvodové zapojení, a i z hlediska parametrů přesnosti nejlepší, se tak jeví již zmíněný senzor PMS5003. Senzor potřebuje pro svou funkci napájení \SI{5}{\volt}, a~komunikuje přes rozhraní UART.

\begin{figure}
    \centering
    \includegraphics[width=0.7\textwidth]{obrazky/PMS5003.jpg}
    \caption{Senzor pro měření koncentrace prachových částic PMS5003. \cite{dat_PMS5003}}
    \label{fig_PMS5003}
\end{figure}

\subsection{Senzor oxidu uhelnatého}

U senzorů oxidu uhelnatého je situace o něco složitější. Na trhu neexistuje mnoho možností, ze kterých by se dalo v rozumné cenové kategorii vybírat. Pokud se podíváme do katalogů většiny (nejen českých) obchodů, zjistíme, že ceny čidel se pohybují ve vyšších stovkách korun až po několik tisíc. Tato cena je pro neprofesionální měření nepřijatelná. Jediným možným rozumným výběrem jsou čidla s označením série MQ od výrobce Hanwei electronics\footnote{\url{https://www.hwsensor.com/}}. Vybral jsem tedy čidlo s označením MQ-7, které je nejvíce citlivé právě na oxid uhelnatý. Funguje na principu ohřevu senzitivní vrstvy (zde konkrétně z materiálu $SnO_{2}$) a následně měření jejího odporu. Senzor je třeba napájet z \SI{5}{\volt} a jeho výstup je třeba přivést na AD převodník, jelikož má analogový výstup.

\begin{figure}
    \centering
    \includegraphics[width=0.7\textwidth]{obrazky/mq7.jpg}
    \caption{Senzor pro měření koncentrace oxidu uhelnatého. \cite{img_mq7}}
    \label{fig_MQ7}
\end{figure}

\subsection{Senzor UV záření}

Na poli relativně levných UV senzorů je výběr opět o něco horší. Existují v podstatě dvě varianty použitelné pro neprofesionální měření a to senzor VEML6075 od výrobce VISHAY a senzor ML8511 výrobce LAPIS Semiconductor. Bohužel první ze zmíněných senzorů se nedá rozumně sehnat, skladové zásoby jsou vyprodané a několik obchodů udává, že se již nevyrábí.

Volím tedy druhý zmíněný senzor ML8511. Tento senzor měří pouze intenzitu UV záření pomocí fotodiody, která je citlivá na UVA a UVB záření. Její největší citlivost je dle datasheetu \cite{dat_ML8511} na vlnovou délku \SI{365}{\nano\metre}. Pro svou činnost potřebuje napájení \SI{3,3}{\volt} a během měření je jeho maximální spotřeba \SI{500}{\micro\ampere}. Pokud jej uvedeme pomocí pinu EN do režimu standby, tak může být spotřeba maximálně \SI{1}{\micro\ampere}. Výstup senzoru je opět napěťový, takže musíme jeho výstup přivést na AD převodník. Rozsah těchto napětí je zhruba od \SI{1}{\volt} do \SI{3}{\volt}, což odpovídá rozsahu \SI{0}{} až \SI{15}{\milli\watt\per\centi\metre\squared}

\subsection{Senzor teploty}

Na poli senzorů pro měření teploty existuje nepřeberné množství různých druhů od spousty výrobců. Pro první základní výběr vhodných senzorů je nutné si definovat alespoň základní parametry a požadavky na takovýto senzor. Vhodnými parametry jsou cena, rozsah měřených teplot (je třeba měřit i při teplotách nižších než \SI{0}{\celsius}), spotřeba a přesnost. Srovnání potenciálně použitelných senzorů se nachází v tabulce \ref{tab_TemperatureSensors}. Cena uvedená v posledním sloupci je brána v jeden den z jednoho e-shopu\footnote{\url{https://www.laskarduino.cz/}}, aby bylo možné objektivně porovnat výsledky mezi sebou.

\begin{table}[]
    \centering
    \begin{tabular}{c|cccc}
    \textbf{Název} & \textbf{Rozsah}                         & \textbf{Přesnost}       & \textbf{Spotřeba}       & \textbf{Cena} \\ \hline
    DS18B20        & \SI{-55}{\celsius} - \SI{125}{\celsius} & $\pm$\SI{0,5}{\celsius} & \SI{1,5}{\milli\ampere} & 68~Kč         \\
    LM75A          & \SI{-25}{\celsius} - \SI{100}{\celsius} & $\pm$\SI{2}{\celsius}   & \SI{280}{\micro\ampere} & 25~Kč         \\
    SHT31          & \SI{-40}{\celsius} - \SI{90}{\celsius}  & $\pm$\SI{0,3}{\celsius} & \SI{350}{\micro\ampere} & 158~Kč        \\
    SHT35          & \SI{-40}{\celsius} - \SI{90}{\celsius}  & $\pm$\SI{0,2}{\celsius} & \SI{1,5}{\milli\ampere} & 378~Kč        \\
    SHT40          & \SI{-40}{\celsius} - \SI{125}{\celsius} & $\pm$\SI{0,2}{\celsius} & \SI{350}{\micro\ampere} & 109~Kč             
    \end{tabular}
    \caption{Srovnání parametrů vybraných senzorů teploty.}
    \label{tab_TemperatureSensors}
\end{table}

Z tabulky \ref{tab_TemperatureSensors} jsem nakonec pro svou práci vybral senzor SHT40 od výrobce SENSIRION. Bohužel není úplně nejlevnější, ale vybral jsem jej díky jeho nízké spotřebě \SI{350}{\micro\ampere} při měření a až \SI{3,4}{\micro\ampere} v nečinném stavu a také relativně vysoké přesnosti, která je až \SI{0,2}{\celsius}. Napájení senzoru je \SI{3,3}{\volt}. S mikroprocesorem senzor komunikuje pomocí sběrnice I$^2$C. Umožňuje měřit i relativní vlhkost vzduchu, jenže při jejím měření se senzor lehce zahřívá a tak je potom měření teploty velice zkreslené, proto tuto jeho funkci nevyužiji a na měření vlhkosti bude použit jiný senzor.

\begin{figure}
    \centering
    \includegraphics[width=0.55\textwidth]{obrazky/sht40.jpg}
    \caption{Senzor pro měření teploty SHT40. \cite{SHT40}}
    \label{fig_SHT40}
\end{figure}

\subsection{Senzor intenzity osvětlení}

Na poli senzorů pro měření intenzity osvětlení existuje hned několik dostupných variant. Liší se převážně způsobem, jakým intenzitu měří a pak také rozsahem, pro které je možné jejich použití. Pro orientační měření lze využít i prostého fotorezistoru, který zapojíme společně s odporem do série a vytvoříme si tak napěťový dělič, na kterém budeme měřiti pomocí AD převodníku analogové napětí. Tento typ měření je však silně závislý na použitém typu fotorezistoru a většinou nebývá moc přesné. Využití této metody je spíše pro účely orientačního měření a určení základních informací a to, jestli je tma či jestli je světlo. 

Dalším z možných typů senzorů je využití fototranzistoru nebo fotodiody. Toto měření je přesnější než předchozí zmíněná metoda, ale vyžaduje znalost přechodové charakteristiky součástky a pro přesnější měření i kalibraci vůči známé referenční hodnotě intenzity osvětlení.

Mnohem vhodnějším typem senzorů pro toto konkrétní použití je tak integrovaný senzor, který obsahuje jednak samotnou na světlo citlivou vrstvu a druhak i řídící logiku, která nám poskytuje digitální výstup ze senzoru například ve formě sériové sběrnice.

V tabulce \ref{tab_LuxIntensitySensors} jsou uvedeny vybrané druhy integrovaných senzorů a jejich základní vlastnosti.

\begin{table}[h]
    \centering
    \begin{tabular}{c|cc}
        \textbf{Název} & \textbf{Rozsah}            & \textbf{Spotřeba}         \\ \hline
        GY-302 BH1750  & \SI{0}{}-\SI{65535}{\lux}  & \SI{200}{\micro\ampere}   \\
        TSL2561        & \SI{0}{}-\SI{40000}{\lux}  & \SI{0,6}{\milli\ampere}   \\
        VEML7700       & \SI{0}{}-\SI{120000}{\lux} & \SI{50}{\micro\ampere}
    \end{tabular}
    \caption{Srovnání parametrů vybraných senzorů intenzity osvětlení}
    \label{tab_LuxIntensitySensors}
\end{table}

Z těchto vybraných dostupných senzorů jsem zvolil poslední z tabulky VEML7700. Tento senzor má velmi nízkou spotřebu i při nejrychlejším cyklu čtení (\SI{100}{\milli\second}) a největší rozsah možného měření. Při porovnání cen se nachází zhruba ve stejné cenové hladině jako druhý nejlepší z této tabulky GY-302~BH1750. S mikrokontrolerem senzor bude komunikovat pomocí sběrnice I$^2$C a napájen bude z \SI{3,3}{\volt}.

\begin{figure}[h!]
    \centering
    \includegraphics[width=0.4\textwidth]{obrazky/veml7700.png}
    \caption{Senzor pro měření intenzity osvětlení VEML7700. \cite{VEML7700}}
    \label{fig_VEML7700}
\end{figure}

\subsection{Senzor atmosférického tlaku}

