\chapter{Návrh zařízení}

Tato kapitola se zabývá návrhem hardwaru celého zařízení. Jedním z~hlavních požadavků je nízká spotřeba, kterou je třeba zohlednit při vybírání použitých senzorů, řídícího mikroprocesoru, komunikačního modulu a~ostatních obvodových komponent. 

\section{Výběr senzorů}
Celé zařízení je schopno měřit koncentraci prachových částic, koncentraci oxidu uhličitého, intenzitu osvětlení, intenzitu UV záření, teplotu, atmosférický tlak a~relativní vlhkost vzduchu. V~následujících částech je popsán výběr z~dostupných senzorů.

\subsection{Senzor koncentrace prachových částic}

Na trhu je dostupných hned několik senzorů na měření koncentrace prachových částic. Jak již bylo zmíněno v~teoretickém úvodu, budu vybírat senzor, který tuto koncentraci určuje na základě osvícení daného vzorku vzduchu a~následně měření odraženého světla. Nyní vyvstává otázka, jestli zvolit senzor, který bude vzorek vzduchu vhánět do měřícího prostoru nuceně pomocí ventilátoru nebo jen za využití stoupání teplého vzduchu. V~tabulce \ref{tab_DustSensors} jsou uvedeny senzory, které jsou relativně cenově přijatelné a~daly by se pro neprofesionální měření využít.

\newcommand{\ugcm}{\micro\gram\per\cubic\meter} % define new command used in '\SI' -> micro gram per cubic meter

\begin{table}[h]
    \caption{Porovnání vybraných parametrů senzorů koncentrace prachových částic.}
    \centering
    \begin{tabular}{c|cccc}
    \textbf{Název}           & \textbf{Rozlišení}                & \textbf{Přesnost}                              & \textbf{Proud}                           & \textbf{Čas čtení}                \\ \hline
    PMS5003 & \multirow{2}{*}{\SI{1}{\ugcm}}    & $0-100$\SI{}{\ugcm}:$\pm$\SI{10}{\micro\gram}  & \multirow{2}{*}{\SI{100}{\milli\ampere}} & \multirow{2}{*}{\SI{10}{\second}} \\
    \cite{dat_PMS5003}                         &                                   & $100-500$\SI{}{\ugcm}:$\pm$\SI{10}{\percent}   &                                          &                                   \\ \hline
    PM1003  & \multirow{2}{*}{\SI{1}{\ugcm}}    & $0-100$\SI{}{\ugcm}:$\pm$\SI{30}{\micro\gram}  & \multirow{2}{*}{\SI{90}{\milli\ampere}}  & \multirow{2}{*}{\SI{30}{\second}} \\
    \cite{PM1003Datasheet} &                                   & $100-500$\SI{}{\ugcm}:$\pm$\SI{30}{\percent}   &                                          &                                   \\ \hline
    PM1006  & \multirow{2}{*}{neuvedeno}        & $0-100$\SI{}{\ugcm}:$\pm$\SI{20}{\micro\gram}  & \multirow{2}{*}{\SI{30}{\milli\ampere}}  & \multirow{2}{*}{\SI{8}{\second}}  \\
    \cite{dat_PM1006} &                                   & $100-500$\SI{}{\ugcm}:$\pm$\SI{20}{\percent}   &                                          &                                   \\ \hline
    GP2Y1010AU0F             & \multirow{2}{*}{\SI{0,5}{\volt}$/$\SI{100}{\ugcm}} & \multirow{2}{*}{záleží na ADC} & \multirow{2}{*}{\SI{20}{\milli\ampere}} & \multirow{2}{*}{\SI{1}{\second}} \\ 
    \cite{dat_GP2Y1010AU0F} & & & & \\ \hline
    \end{tabular}
    \label{tab_DustSensors}
\end{table}

Po pečlivém prostudování jednotlivých parametrů jsem zvolil senzor PMS5003 od firmy PLANTOWER. Důležitým aspektem při výběru byla také cena tohoto senzoru, v~době vypracovávání této práce jej šlo pořídit za zhruba 350~Kč. Dalším důležitým parametrem byla spotřeba proudu v~aktivním stavu. Na první pohled se může zdát, že oproti všem senzorům má spotřebu nejvyšší. Oproti PM1003 má však třetinový čas potřebný k~získání měřených dat, takže spotřebovává sice vyšší proud, ale po kratší časový úsek. PM1006 má spotřebu proudu zhruba třetinovou, ale vzorek měřeného vzduchu se do měřícího prostoru dostává pomocí sálání a~tak je nutné zajistit konstrukčně dostatečně a~správně dimenzované průduchy a~také konstantní polohu a~hlavně náklon senzoru, což by mohlo být v~praxi téměř nemožné, pokud má být zařízení používáno také ve volném prostranství. Poslední ze senzorů v~tabulce \ref{tab_DustSensors} má zdánlivě nejlepší výsledky. Bohužel se jedná pouze o~měřící modul samotný, který neobsahuje žádný ventilátor ani řídící logiku, je tedy třeba tyto věci zapojit a~konstrukčně vyřešit. Nejjednodušší na obvodové zapojení, a~i~z~hlediska parametrů přesnosti nejlepší, se tak jeví již zmíněný senzor PMS5003. Senzor potřebuje pro svou funkci napájení \SI{5}{\volt} a~komunikuje přes rozhraní UART.

\begin{figure}
    \centering
    \includegraphics[width=0.7\textwidth]{obrazky/PMS5003.jpg}
    \caption[Senzor pro měření koncentrace prachových částic PMS5003.]{Senzor pro měření koncentrace prachových částic PMS5003 \cite{dat_PMS5003}.}
    \label{fig_PMS5003}
\end{figure}

\subsection{Senzor oxidu uhličitého}

U senzorů oxidu uhličitého je situace o~něco složitější. Na trhu neexistuje mnoho možností, ze kterých by se dalo v~rozumné cenové kategorii vybrat čidlo pro amatérské použití. Většina těchto čidel je založena na optickém principu a~jejich cena začíná na jednotkách tisíc korun. Tato cena je pro neprofesionální měření neakceptovatelná. Jedním z~mála senzorů dostupných pro amatérské účely jsou senzory od firmy Sensirion\footnote{\url{https://sensirion.com/}} a~to konkrétně senzor SGP30. Tento senzor pracuje na elektrochemickém principu a~umožňuje měřit ekvivalentní hodnotu CO$_2$ a~také hodnotu TVOC (Total Volatile Organic Compounds). Senzor potřebuje pro svou funkci napájení~z~\SI{1.8}{\volt} a~komunikuje s~řídícím mikrokontrolerem skrze sběrnici I2C.

% U~senzorů oxidu uhelnatého je situace o~něco složitější. Na trhu neexistuje mnoho možností, ze kterých by se dalo v~rozumné cenové kategorii vybírat. Pokud se podíváme do katalogů většiny (nejen českých) obchodů, zjistíme, že ceny čidel se pohybují ve vyšších stovkách korun až po několik tisíc. Tato cena je pro neprofesionální měření nepřijatelná. Jediným možným rozumným výběrem jsou čidla s~označením série MQ od výrobce Hanwei electronics\footnote{\url{https://www.hwsensor.com/}}. Vybral jsem tedy čidlo s~označením MQ-7, které je nejvíce citlivé právě na oxid uhelnatý. Funguje na principu ohřevu senzitivní vrstvy (zde konkrétně z~materiálu $SnO_{2}$) a~následně měření jejího odporu. Senzor je třeba napájet z~\SI{5}{\volt} a~jeho výstup je třeba přivést na AD převodník, jelikož má analogový výstup.

% \begin{figure}
%     \centering
%     \includegraphics[width=0.7\textwidth]{obrazky/mq7.jpg}
%     \caption[Senzor pro měření koncentrace oxidu uhelnatého.]{Senzor pro měření koncentrace oxidu uhelnatého \cite{img_mq7}.}
%     \label{fig_MQ7}
% \end{figure}

\subsection{Senzor UV záření}

Na poli relativně levných UV senzorů je výběr opět o~něco horší. Existují v~podstatě dvě varianty použitelné pro neprofesionální měření a~to senzor VEML6075 od výrobce VISHAY a~senzor ML8511 výrobce LAPIS Semiconductor. Bohužel první ze zmíněných senzorů se nedá rozumně sehnat, skladové zásoby jsou vyprodané a~několik obchodů udává, že se již nevyrábí.

Byl proto zvolen druhý senzor ML8511. Tento senzor měří pouze intenzitu UV záření pomocí fotodiody, která je citlivá na UVA a~UVB záření. Její největší citlivost je dle katalogového listu \cite{dat_ML8511} na vlnovou délku \SI{365}{\nano\metre}. Pro svou činnost potřebuje napájení \SI{3,3}{\volt} a~během měření je jeho maximální spotřeba \SI{500}{\micro\ampere}. Pokud jej uvedeme pomocí pinu EN do režimu standby, tak může být spotřeba maximálně \SI{1}{\micro\ampere}. Výstup senzoru je opět napěťový, takže musíme jeho výstup přivést na AD převodník. Rozsah těchto napětí je zhruba od \SI{1}{\volt} do \SI{3}{\volt}, což odpovídá rozsahu \SI{0}{} až \SI{15}{\milli\watt\per\centi\metre\squared}.

\begin{figure}[h]
    \centering
    \includegraphics[width=0.25\textwidth]{obrazky/ml8511.jpg}
    \caption{Senzor pro měření intenzity UV záření ML8511 \cite{picture_ML8511}.}
    \label{fig_ml8511}
\end{figure}

\subsection{Senzor teploty}

Na poli senzorů pro měření teploty existuje nepřeberné množství různých druhů od spousty výrobců. Pro první základní výběr vhodných senzorů je nutné si definovat alespoň základní parametry a~požadavky na takovýto senzor. Vhodnými parametry jsou cena, rozsah měřených teplot (je třeba měřit i~při teplotách nižších než \SI{0}{\celsius}), spotřeba a~přesnost. Srovnání potenciálně použitelných senzorů se nachází v~tabulce~\ref{tab_TemperatureSensors}. Cena uvedená v~posledním sloupci je brána v~jeden den z~jednoho e-shopu\footnote{\url{https://www.laskarduino.cz/}}, aby bylo možné objektivně porovnat výsledky mezi sebou.

\begin{table}[b]
    \caption{Srovnání parametrů vybraných senzorů teploty.}
    \centering
    \begin{tabular}{c|cccc}
    \textbf{Název} & \textbf{Rozsah}                         & \textbf{Přesnost}       & \textbf{Spotřeba}       & \textbf{Cena} \\ \hline
    DS18B20 \cite{dat_DS18B20}        & \SI{-55}{\celsius} až \SI{125}{\celsius} & $\pm$\SI{0,5}{\celsius} & \SI{1,5}{\milli\ampere} & 68~Kč         \\
    LM75A \cite{dat_LM75A}          & \SI{-25}{\celsius} až \SI{100}{\celsius} & $\pm$\SI{2}{\celsius}   & \SI{280}{\micro\ampere} & 25~Kč         \\
    SHT31 \cite{dat_SHT31}          & \SI{-40}{\celsius} až \SI{90}{\celsius}  & $\pm$\SI{0,3}{\celsius} & \SI{350}{\micro\ampere} & 158~Kč        \\
    SHT35 \cite{dat_SHT35}          & \SI{-40}{\celsius} až \SI{90}{\celsius}  & $\pm$\SI{0,2}{\celsius} & \SI{1,5}{\milli\ampere} & 378~Kč        \\
    SHT40 \cite{dat_SHT40}          & \SI{-40}{\celsius} až \SI{125}{\celsius} & $\pm$\SI{0,2}{\celsius} & \SI{350}{\micro\ampere} & 109~Kč             
    \end{tabular}
    \label{tab_TemperatureSensors}
\end{table}

Z tabulky \ref{tab_TemperatureSensors} jsem nakonec pro svou práci vybral senzor SHT40 od výrobce SENSIRION. Bohužel není úplně nejlevnější, ale vybral jsem jej díky jeho nízké spotřebě \SI{350}{\micro\ampere} při měření a~až \SI{3,4}{\micro\ampere} v~nečinném stavu a~také relativně vysoké přesnosti, která je až \SI{0,2}{\celsius}. Napájení senzoru je \SI{3,3}{\volt}. S~mikroprocesorem senzor komunikuje pomocí sběrnice I$^2$C. 

\begin{figure}
    \centering
    \includegraphics[width=0.55\textwidth]{obrazky/sht40.jpg}
    \caption[Senzor pro měření teploty SHT40.]{Senzor pro měření teploty SHT40 \cite{SHT40}.}
    \label{fig_SHT40}
\end{figure}

\subsection{Senzor intenzity osvětlení}

Na poli senzorů pro měření intenzity osvětlení existuje hned několik dostupných variant. Liší se převážně způsobem, jakým intenzitu měří a~pak také rozsahem, pro které je možné jejich použití. Pro orientační měření lze využít i~prostého fotorezistoru, který zapojíme společně s~odporem do série a~vytvoříme si tak napěťový dělič, na kterém budeme měřit pomocí AD převodníku analogové napětí. Tento typ měření je však silně závislý na použitém typu fotorezistoru a~většinou nebývá moc přesné. Využití této metody je spíše pro účely orientačního měření a~určení základních informací a~to, jestli je tma či jestli je světlo. 

Dalším z~možných typů senzorů je využití fototranzistoru nebo fotodiody. Toto měření je přesnější než předchozí zmíněná metoda, ale vyžaduje znalost přechodové charakteristiky součástky a~pro přesnější měření i~kalibraci vůči známé referenční hodnotě intenzity osvětlení.

Mnohem vhodnějším typem senzorů pro toto konkrétní použití je tak integrovaný senzor, který obsahuje nejen samotnou na světlo citlivou vrstvu, ale také i~řídící logiku, která nám poskytuje digitální výstup ze senzoru například ve formě sériové sběrnice. V~tabulce \ref{tab_LuxIntensitySensors} jsou uvedeny vybrané druhy integrovaných senzorů a~jejich základní vlastnosti.

\begin{table}[h]
    \caption{Srovnání parametrů vybraných senzorů intenzity osvětlení.}
    \centering
    \begin{tabular}{c|cc}
        \textbf{Název} & \textbf{Rozsah}            & \textbf{Spotřeba}         \\ \hline
        GY-302 BH1750 \cite{dat_BH1750}  & \SI{0}{}-\SI{65535}{\lux}  & \SI{200}{\micro\ampere}   \\
        TSL2561 \cite{dat_TSL2561}       & \SI{0}{}-\SI{40000}{\lux}  & \SI{0,6}{\milli\ampere}   \\
        VEML7700 \cite{dat_VEML7700}       & \SI{0}{}-\SI{120000}{\lux} & \SI{50}{\micro\ampere}
    \end{tabular}
    \label{tab_LuxIntensitySensors}
\end{table}

Z těchto vybraných dostupných senzorů jsem zvolil poslední z~tabulky VEML7700. Tento senzor má velmi nízkou spotřebu i~při nejrychlejším cyklu čtení (\SI{100}{\milli\second}) a~největší rozsah možného měření. Při porovnání cen se nachází zhruba ve stejné cenové hladině jako druhý nejlepší z~této tabulky GY-302~BH1750. S~mikrokontrolerem senzor bude komunikovat pomocí sběrnice I$^2$C a~napájen bude z~\SI{3,3}{\volt}.

\begin{figure}[h!]
    \centering
    \includegraphics[width=0.4\textwidth]{obrazky/veml7700.png}
    \caption[Senzor pro měření intenzity osvětlení VEML7700.]{Senzor pro měření intenzity osvětlení VEML7700 \cite{VEML7700}.}
    \label{fig_VEML7700}
\end{figure}

\subsection{Senzor atmosférického tlaku}

Posledním z~potřebných senzorů je senzor pro měření atmosférického tlaku. Existuje několik senzorů, které integrují do jednoho pouzdra měření teploty, vlhkosti i~atmosférického tlaku. Tyto senzory se však vyznačují nižší přesností. Jedním z~takovýchto senzorů je například BME280 od výrobce Bosch, který je populární mezi kutily při stavění amatérské domácí meteostanice. Bohužel poslední dobou není dostupný skladem v~žádném z~velkých obchodů a~pokud se někde vyskytne, stojí několikanásobek jeho normální ceny a~je tak pro tuto práci nepoužitelný.

Budu tedy porovnávat senzory, které měří pouze atmosférický tlak a~jsou dostupné a~mají relativně nízkou cenu.

\begin{table}[h]
    \caption{Srovnání parametrů vybraných senzorů atmosférického tlaku.}
    \centering
    \begin{tabular}{c|ccc}
        \textbf{Název} & \textbf{Rozsah}                     & \textbf{Přesnost}      & \textbf{Spotřeba}        \\ \hline
        BMP180 \cite{dat_BMP180}         & \SI{300}{}-\SI{1100}{\hecto\pascal} & $\pm$\SI{3}{\pascal}   & \SI{12}{\micro\ampere}   \\
        BMP280 \cite{dat_BMP280}         & \SI{300}{}-\SI{1100}{\hecto\pascal} & $\pm$\SI{1,3}{\pascal} & \SI{2,7}{\micro\ampere}  \\
        BMP388 \cite{dat_BMP388}         & \SI{300}{}-\SI{1250}{\hecto\pascal} & $\pm$\SI{8}{\pascal}   & \SI{3,4}{\micro\ampere}  \\
        BME280 \cite{dat_BME280}         & \SI{300}{}-\SI{1100}{\hecto\pascal} & $\pm$\SI{2}{\pascal}   & \SI{7,1}{\micro\ampere}  \\
        ICP-10100 \cite{dat_ICP-10100}      & \SI{300}{}-\SI{1150}{\hecto\pascal} & $\pm$\SI{3,2}{\pascal} & \SI{10,4}{\micro\ampere} 
    \end{tabular}
    \label{tab_AirPressureSensors}
\end{table}

Z~tabulky \ref{tab_AirPressureSensors}, je vidět, že většina senzorů atmosférického tlaku je od výrobce Bosch Sensortec. Vyskytuje se zde již zmiňovaný BME280, který je ale moc drahý a~momentálně nedostupný a~při měření tlaku má jednu z~vyšších spotřeb. Dalším ideálním adeptem by byl i~senzor BMP280, který umožňuje měřit atmosférický tlak i~teplotu. Bohužel ani tento senzor není příliš dostupný a~dle oficiálních stránek výrobce již není doporučen pro nové návrhy.

Vybral jsem tedy senzor BMP388 od firmy Bosch Sensortec. Nepatří bohužel mezi nejlevnější, ale zato je dostupný v~obchodech a~poskytuje vzhledem ke své dostupnosti relativně dobré parametry samotného měření. Stejně jako dříve zmíněné senzory, i~tento umožňuje kromě atmosférického tlaku měřit i~teplotu. Pro komunikaci s~mikrokontrolerem lze použít sběrnici I$^2$C nebo SPI, jelikož je většina vybraných senzorů na sběrnici I$^2$C, použiji stejnou sběrnici i~pro tento senzor. Pro napájení je třeba přivést \SI{3,3}{\volt}.

\subsection{Senzor měření vlhkosti}

Jak již bylo zmíněno v~kapitole o~výběru senzoru pro měření teploty, vybraný senzor SHT40 umožňuje měřit i~vzdušnou vlhkost. Rozsah měření je \SI{0}{}-\SI{100}{\percent} relativní vlhkosti s~přesností $\pm$\SI{1,8}{\percent}. Pro potřeby měření relativní vzdušné vlhkosti v~této práci jsou tyto parametry s~přehledem dostatečné.

\section{Výběr řídícího mikrokontroléru}

Mikrokontroler je hlavním řídícím prvkem celého zařízení. Při jeho výběru je nutné dbát na spousty mnohdy protichůdných parametrů. Jedním z~hlavních a~nejdůležitějších parametrů jsou požadavky na hardwarové periferie a~celkově výbavu daného mikrokontroleru. Jak již bylo popsáno v~předchozích kapitolách této práce, je nutné všechny senzory připojit pomocí různých sběrnic či zajistit AD převodník pro připojení analogových výstupů ze senzorů. Také je třeba dbát na dostatečný počet vstupně-výstupních pinů.

V dnešní době má spousta mikrokontrolerů přímo v~sobě integrovanou rádiovou část, takže jsou schopny se připojit např. na WiFi, komunikovat s~ostatními zařízeními přes Bluetooth či posílat zprávy přes LoRa síť. Toto řešení zjednodušuje návrh výsledného zařízení a~také dokáže snížit výrobní náklady, jelikož je vše obsaženo v~jednom čipu a~není třeba osazovat několik samostatných čipů. Zároveň snižuje pravděpodobnost chybného návrhu nebo eliminuje další možný zdroj poruch, jelikož každý další použitý čip na desce a~spojení k~němu je možným zdrojem problémů.

Pro výběr v~této práci budu uvažovat výběr mikrokontrolerů od největších výrobců jako jsou STMicroelectronics, Atmel (dnes Microchip) nebo Espressif Systems. Existuje samozřejmě spousta dalších výrobců, ale tihle uvedení jsou jedni z~největších, nejznámějších a~nejdostupnějších. Porovnání ceny v~tabulce \ref{tab_MCU} je provedeno v~jeden den z~jednoho obchodu\footnote{\url{https://www.tme.eu/cz/}} pro možnost objektivního posouzení. Do tabulky pro srovnání jsem vybral pouze nejdůležitější parametry daných mikrokontrolerů jako jsou hardwarové periferie pro sběrnice, počet GPIO (vstupně-výstupních pinů), přítomnost WiFi rozhraní a~cenu.

\begin{table}[h]
    \caption{Srovnání parametrů vybraných mikrokontrolerů.}
    \centering
    \begin{tabular}{c|cccccc}
        \textbf{Název}                         & \textbf{I$^2$C} & \textbf{SPI} & \textbf{GPIO} & \textbf{USART} & \textbf{WiFi} & \textbf{Cena} \\ \hline
        ATmega328P \cite{dat_ATmega328p}       & 1 & 1 & 23 & 1 & Ne  & 70~Kč \\
        ESP32 WROOM \cite{dat_ESP32-WROOM}     & 2 & 4 & 34 & 3 & Ano & 80~Kč \\
        ATSAM4LC2 \cite{dat_ATSAM4LC2}         & 2 & 1 & 27 & 3 & Ne  & 90~Kč \\
        STM8L162R8T6 \cite{dat_STM8L162R8T6}   & 1 & 2 & 54 & 3 & Ne  & 110~Kč \\
        STM32L051C8T6 \cite{dat_STM32L051C8T6} & 2 & 2 & 37 & 2 & Ne  & 180~Kč
        
    \end{tabular}
    \label{tab_MCU}
\end{table}

Z výše uvedené tabulky \ref{tab_MCU} jsem nakonec vybral mikrokontroler ESP32~WROOM od výrobce Espressif Systems. Z~hlediska ceny není nejlevnější, ale pokud se podíváme na parametry, které za tuto cenu nabízí, tak je to bezkonkurenční nabídka. Mikrokontroler samotný obsahuje kromě výše zmíněných parametrů i~rádiovou část ve které je obsažena WiFi a~Bluetooth, takže pro bezdrátové spojení není třeba použít žádný další modul. Mikrokontroler pracuje až na frekvenci \SI{240}{\mega\hertz} a~obsahuje dvě jádra, takže je možné velmi rychle paralelně zpracovávat data. Mikrokontroler samotný je navržen pro IoT aplikace, takže je dbáno na velmi nízkou spotřebu při práci i~v~mnoha režimech spánku, které je možné aktivovat. Hlavní dvě jádra mikrokontroleru jsou doplněna o~tzv. ULP (Ultra Low Power) koprocesor, který je možné aktivovat v~režimu spánku a~vykonávat tak velice jednoduché sekvence příkazů a~ovládat například výstupní piny. Na obrázku \ref{fig_ESP32InternalStructure} je vidět blokové schéma struktury mikroprocesoru ESP32 včetně všech jeho hardwarových periferií. Celý mikrokontroler je třeba napájet napětím \SI{3,3}{\volt} a~zdroj musí být schopen dodat alespoň \SI{500}{\milli\ampere}, tato hodnota je relativně vysoká, ale je to způsobeno potřebou vyššího příkonu při vysílání přes WiFi.

\begin{figure}
    \centering
    \includegraphics[width=0.7\textwidth]{obrazky/esp32_internalStructure.png}
    \caption[Blokový diagram mikrokontroleru ESP32.]{Blokový diagram mikrokontroleru ESP32 \cite{dat_ESP32-WROOM}.}
    \label{fig_ESP32InternalStructure}
\end{figure}

\subsection{Analogově digitální převodník}

Jak již bylo zmíněno v~předchozích kapitolách o~výběru senzorů, bude potřeba zajistit analogové vstupy na mikrokontroleru. Bohužel ESP32 nemá příliš přesný AD převodník a~je třeba provádět poměrně náročnou kalibraci pro každý zakoupený čip zvlášť, jak je zmíněno přímo v~oficiální dokumentaci\footnote{\url{https://docs.espressif.com/projects/esp-idf/en/latest/esp32/api-reference/peripherals/adc.html}}. Nelze provádět ani kalibraci na jednom vzorku pro konkrétní výrobní sérii.

Kvůli těmto výše uvedeným důvodům jsem se rozhodl pro tuto práci využít externí AD převodník, čímž by se měla zvýšit přesnost měření a~zajistit reprodukovatelné výsledky při použití jiného mikrokontroleru. Hlavním požadavkem na výběr tohoto převodníku byla cena, dostupnost a~spotřeba. Pro tuto práci je zapotřebí aby byl napájen z~\SI{3,3}{\volt} a~měl alespoň 2 vstupní kanály. Těmto požadavkům nejlépe vyhověl AD převodník MCP3202 od firmy Microchip Technology Inc. \cite{dat_MCP3202}. Obsahuje dva kanály s~rozlišením 12 bitů. Připojení k~mikrokontroleru je provedeno přes sběrnici SPI a~maximální spotřeba při čtení je zhruba \SI{320}{\micro\ampere}.

\section{Přenos dat na server}

Pro přenos naměřených dat na server je k~dispozici celá řada možností, jak to provést. Jelikož budou data zpracovávána na serveru, který je dostupný přes síť internet, je třeba zajistit, aby tam data byla poslána. Pro první pokusy bude nejjednodušší použití sítě WiFi, do které se lze se zařízením jednoduše připojit a~poté se také připojit na aplikační server. Toto řešení je ovšem nevhodné pro použití kdekoliv mimo obydlené oblasti či oblasti, kde máme pokrytí svou vlastní WiFi, protože se nelze spoléhat na to, že v~daném místě potřeby bude nějaká např. veřejná síť. Další z~nevýhod této technologie je její relativní energetická náročnost, jelikož je třeba vysílat na frekvenci \SI{2,4}{\giga\hertz} a~po připojení musí zařízení získat IP adresu, což nějakou dobu trvá a~poté může teprve probíhat komunikace. Jelikož však síť internetu a~její protokoly nejsou uzpůsobeny na redukci datového toku, je celý přenos výrazně delší než vyslání zprávy např. přes síť LoRaWAN, což má opět negativní dopad na spotřebu energie.

Z výše uvedených důvodů se hodí využití jiné sítě, která je pro tato IoT zařízení přizpůsobená, je méně energeticky náročná a~umožňuje připojení zařízení na řádově větší vzdálenosti. Jako nejjednodušší se jeví použití sítě LoRa. Pro tuto síť existuje na trhu mnoho komunikačních modulů, které jsou i~relativně cenově přijatelné. Síť jako taková není zatížena licenčními poplatky. Je zde však možnost využití již existující infrastruktury od nějaké firmy (u nás například České Radiokomunikace a.s.\footnote{https://www.cra.cz/pripojeni-k-iot-siti-lorawan}), kde se poté platí poplatky za využívání připojení k~jejich síti či případné další služby.

Pokud chceme připojit zařízení přes LoRaWAN, ale nechceme být zavázání poskytovateli služeb a~platit poplatky, je možnost použít některou z~dostupných sítí, které jsou zdarma. Většinou se zde vyskytují omezení např. v~počtu přenesených zpráv za daný čas nebo počet připojených zařízení. Jednou z~nejznámějších sítí je TTN (The Things Network\footnote{https://www.thethingsnetwork.org/}). Princip této sítě je postaven na infrastruktuře, kterou do sítě připojují samotní uživatelé a~tato služba je koncentruje na jeden server. Uživatelé tak mají možnost připojovat svoje vlastní gateway a~poté libovolně v~dosahu jakékoli jiné gateway připojené do sítě přenášet data ze svých zařízení.

\subsection{Výběr LoRa modulu}

Pro výběr LoRa modulu bude nejdůležitějším parametrem cena a~dostupnost. Na našem území je dle ČTÚ a~její národní kmitočtové tabulky \cite{kmitoctovaTabulka} povoleno provozovat LoRa zařízení v~ISM pásmu \SI{868}{MHz}. Toto je také třeba zohlednit při výběru vhodného modulu. Dalším z~kritérií byla dostupnost kvalitní dokumentace a~také jestli existují příklady k~použití daného modulu.

\DeclareSIUnit \belm {Bm}   %% declaration of Bel upper the milliwatt unit
Daným kritériím bezesporu vyhověl modul s~označením RFM95W \cite{dat_RFM95W}, založený na čipu RF96. Jak již bylo zmíněno, modul pracuje na frekvenci \SI{868}{\mega\hertz} a~nejvyšší možný výkon vysílače je \SI{20}{\deci\belm}. Modul pro své fungování potřebuje napájení \SI{3,3}{\volt} a~při běžném vysílání výkonem \SI{+7}{\deci\belm} a~dokonale přizpůsobené anténě na impedanci \SI{50}{\ohm} je jeho spotřeba \SI{20}{\milli\ampere}. Pro připojení k~mikrokontroleru je zapotřebí sběrnice SPI a~alespoň 1 GPIO (ideálně s~podporou přerušení), jelikož modul umožňuje při přijetí dat ze sítě změnit logickou hodnotu na tomto pinu a~tím dát hlavnímu mikrokontroleru vědět, že má přijatá data přečíst a~zpracovat. 

\begin{figure}[h]
    \centering
    \includegraphics[width=0.7\textwidth]{obrazky/rfm95w.jpg}
    \caption{LoRa modul RFM95W.}
    \label{fig_RFM95W}
\end{figure}

\section{Napájení zařízení}

Celé zařízení ke své funkci potřebuje primárně napětí \SI{3,3}{\volt}. Toto napětí bude bráno přímo z~baterie LiFePO4, která má při plném nabití \SI{3,6}{\volt}. Vyšší napětí ničemu nevadí, jelikož všechny použité senzory a~komponenty mají dovolené napětí mezi \SI{3}{\volt} až právě \SI{3,6}{\volt}. Toto napětí tedy bude přivedeno do zařízení přes tranzistor jako ochrana proti přepólování baterie, spínací tranzistor pro sepnutí celého zařízení a~pojistku. Tímto v~podstatě přímým připojením mikrokontroleru a~jednotlivých senzorů na baterii může být dosaženo výrazné úspory energie při režimu spánku. Lineární stabilizátory i~spínané měniče pracují s~jistou účinností, ta je ovšem většinou dána pro nějaký nominální proud, ale při minimálním proudovém odběru (režim spánku) se projevuje jejich samotná spotřeba, která může být několikanásobně vyšší než spotřeba mikrokontroleru v~režimu spánku.

Jelikož je použit senzor prachových částic PMS5003, který potřebuje pro svou činnost napětí \SI{5}{\volt}, je třeba použití zvyšujícího měniče (step-up). Dále je třeba zajistit stabilní napětí \SI{3,3}{\volt} pro externí AD převodník, zde nelze použít napětí přímo~z~baterie, jelikož by kolísalo referenční napětí a~bylo by tudíž nemožné cokoliv změřit. Dále je vhodné zajistit pro AD převodník napájecí napětí, které nebude zatíženo zvlněním či jinými neduhy danými spínaným měničem. Z~tohoto důvodu bude pro jeho napájení použit lineární stabilizátor. Protože je však rozdíl mezi vstupním napětím (\SI{5}{\volt}) a~potřebným výstupním napětím (\SI{3,3}{\volt}) nižší než \SI{2}{\volt}, je třeba použít tzv. LDO (Low Dropout) lineární stabilizátor. Poslední z~potřebných napěťových úrovní je \SI{1.8}{\volt}. Toto napětí slouží pro napájení senzoru CO$_2$ a~i~zde bude nejvhodnější volbou LDO stabilizátor, jelikož senzor potřebuje pro svou funkci maximálně \SI{48.8}{\milli\ampere}.

\subsection{Výběr step-up měniče}

Step-up měničů existují na trhu stovky různých druhů od všemožných výrobců. Výběr je tedy nutné provést převážně na základě potřebných parametrů jako jsou vstupní napětí, výstupní napětí a~potřebný dodávaný proud. Jelikož je celé zařízení koncipováno jako nízkopříkonové, je vhodné podívat se také na výrobcem udávaný proud potřebný pro provoz samotného měniče. Nejdůležitějším parametrem je bohužel v~dnešní době dostupnost daného měniče a~také jeho cena.

\begin{figure}[h]
    \centering
    \includegraphics[width=0.7\textwidth]{obrazky/schematicAP1603.png}
    \caption{Typické zapojení step-up měniče AP1603 pro výstupní napětí \SI{5}{\volt} \cite{dat_AP1603}.}
    \label{fig_schematic-AP1603}
\end{figure}

Pro tuto práci byl vybrán step-up měnič AP1603 od výrobce Diodes Incorporated \cite{dat_AP1603}. Měnič dokáže fungovat od napětí \SI{0.9}{\volt} až do \SI{5.5}{\volt} a~poskytovat výstupní proud až \SI{200}{\milli\ampere}. Díky tomu, že měnič obsahuje spínací tranzistory přímo na čipu, tak není potřeba pro funkci příliš mnoho externích součástek. Na obrázku \ref{fig_schematic-AP1603} je vidět zapojení pro výstupní napětí \SI{5}{\volt}. 

Při návrhu hodnot použitích součástek v~zapojení step-up měniče budu vycházet z~katalogového listu poskytnutého výrobcem \cite{dat_AP1603}. Je potřeba připojit k~měniči pouze vstupní a~výstupní elektrolytické filtrační kondenzátory o~hodnotě \SI{47}{\micro\farad}, filtrační kondenzátor o~hodnotě \SI{100}{\nano\farad} na referenční pin měniče a~dále spínací tlumivku velikosti \SI{22}{\micro\henry} s~dostatečným proudovým dimenzováním. U~tohoto měniče není potřeba žádný napěťový dělič ve zpětné vazbě, jelikož výrobce uvažoval použití s~výstupem \SI{5}{\volt}, stačí tedy zpětnou vazbu připojit přímo na výstup měniče. Výsledné zapojení je vidět na obrázku \ref{fig_StepUp-kicad}.

\begin{figure}
    \centering
    \includegraphics[scale=1.2]{obrazky/stepup_schematic.pdf}
    \caption{Schéma zapojení step-up měniče.}
    \label{fig_StepUp-kicad}
\end{figure}


% \subsection{Výběr step-down měniče}

% Step-down měničů existují na trhu stovky různých druhů od všemožných výrobců. Výběr je tedy nutné provést převážně na základě potřebných parametrů jako jsou vstupní napětí, výstupní napětí a~potřebný dodávaný proud. Jelikož je celé zařízení koncipováno jako nízkopříkonové, je vhodné podívat se také na výrobcem udávaný proud potřebný pro provoz samotného měniče. nejdůležitějším parametrem je bohužel v~dnešní době dostupnost daného měniče a~také jeho cena.

% Já jsem pro svou práci vybral step-down měnič TPS54331 od výrobce Texas Instruments Incorporated\cite{dat_TPS54331}. Tento měnič lze používat až do vstupního napětí \SI{28}{\volt} a~odebírat z~něj výstupní proud \SI{3}{\ampere}. Díky tomu, že měnič obsahuje výkonový spínací tranzistor, tak není potřeba příliš mnoho externích součástek. Na obrázku \ref{fig_Schematic-TPS54331} je vidět zjednodušené zapojení tohoto měniče.

% \begin{figure}
%     \centering
%     \includegraphics[width=0.6\textwidth]{obrazky/schematicTPS54331.png}
%     \caption[Zjednodušené schéma zapojení měniče TPS54331.]{Zjednodušené schéma zapojení měniče TPS54331 \cite{dat_TPS54331}.}
%     \label{fig_Schematic-TPS54331}
% \end{figure}

% \subsubsection{Návrh hodnot součástek step-down měniče}

% Při návrhu součástek budu vycházet z~katalogového listu poskytnutého výrobcem \cite{dat_TPS54331}. Prvním z~potřebných kroků je zvolení odporů do napěťového děliče tvořícího zpětnou vazbu, která zajišťuje nastavení výstupního napětí. Jejich hodnotu spočítám z~rovnice \ref{eq_VoltageDivider}.
% \begin{equation}
%     R_{O2}=\frac{R_{O1}\cdot V_{ref}}{V_{out}-V_{ref}}
%     \label{eq_VoltageDivider}
% \end{equation}

% Kde známe hodnoty $V_{ref} = \SI{0,8}{\volt}$ a~hodnotu požadovaného výstupního napětí $V_{out} = \SI{3,3}{\volt}$. Z~toho vyplývá, že máme dva neznámé odpory a~musíme jeden z~nich zvolit. Já si volím například $R_{O1}=\SI{15}{\kilo\ohm}$ a~poté dopočítám $R_{O2}$ jako:

% \begin{equation}
%     R_{O2}=\frac{R_{O1}\cdot V_{ref}}{V_{out}-V_{ref}}=\frac{15000\cdot 0,8}{3,3-0,8}=\SI{4800}{\ohm}
%     \label{eq_VoltageDivider-full}
% \end{equation}

% Tento odpor bohužel není ve standardní řadě odporů, proto jsem jej zaokrouhlil na nejbližší hodnotu $R_{O2}=\SI{4,7}{\kilo\ohm}$, v~tomto případě bude výstupní napětí $V_{out}=\SI{3,35}{\volt}$, což je naprosto přijatelné, alespoň bude menší rezerva při poklesu napětí se skokovou změnou zátěže.

% Další z~potřebných volených součástek jsou vstupní filtrační kondenzátory, ve schématu na obrázku \ref{fig_Schematic-TPS54331} označovány jako $C1$. Dle katalogového listu je vhodné zvolit hodnotu okolo \SI{10}{\micro\farad}. Já zvolím 2x \SI{4,7}{\micro\farad} keramický kondenzátor z~dielektrika X5R na napětí \SI{16}{\volt} a~k~těmto dvěma ještě jeden \SI{10}{\nano\farad} pro potlačení vysokofrekvenčního rušení.

% Poslední z~hlavních počítaných součástek step-down měniče je výstupní LC filtr. Hlavní součástkou tohoto filtru je cívka, která musí mít správnou velikost indukčnosti i~z~pohledu fungování měniče a~také musí být dostatečně proudově dimenzovaná podle odebíraného výstupního proudu. Minimální hodnota indukčnosti se spočítá z~rovnice \ref{eq_Inductor} kde $V_{OUT(max)}=V_{OUT}=\SI{3,3}{\volt}$, $V_{IN(max)}=\SI{5}{\volt}$, $I_{OUT}=\SI{1,5}{\ampere}$ a~$f_{SW}=\SI{570}{\kilo\hertz}$. Dále je zde přítomen parametr $K_{IND}$, který se volí podle druhu použitých výstupních kondenzátorů. Já budu mít keramické kondenzátory a~ty mají nízké ESR, takže mohu tento parametr volit jako $K_{IND}=0.3$.

% \begin{equation}
%     L_{MIN}=\frac{V_{OUT(max)}\cdot (V_{IN(max)}-V_{OUT})}{V_{IN(max)}\cdot K_{IND}\cdot I_{OUT}\cdot f_{SW}}=\frac{3.3\cdot (5-3.3)}{5\cdot 0.3\cdot 1.5 \cdot 570000}=\SI{4,37}{\micro\henry}
%     \label{eq_Inductor}
% \end{equation}

% Tato vypočítaná hodnota je minimální a~je třeba zvolit alespoň nejbližší vyšší běžně dostupnou hodnotu, takže výsledná hodnota indukčnosti cívky bude $L=\SI{4,7}{\micro\henry}$. 

% Hodnotu výstupních filtračních kondenzátorů jsem zvolil podle tabulky~1 v~katalogovém listu \cite{dat_TPS54331} a~vychází nám z~toho 2x \SI{47}{\micro\farad}. Z~této stejné tabulky jsem určil také hodnoty součástek kompenzační smyčky $C1$,$C2$ a~$R3$.

% Nedílnou součástí měniče je dioda na výstupu $D1$, kterou je třeba dostatečně proudově a~napěťově dimenzovat. Je vhodné také zvolit diodu, která má nízký úbytek napětí (Schottkyho dioda). Já jsem zvolil diodu s~označením SS54 od výrobce Microdiode Electronics. Maximální napětí na této diodě je \SI{40}{\volt}, maximální procházející proud je \SI{5}{\ampere} a~úbytek na této diodě je \SI{550}{\milli\volt}.

% Výsledné zapojení tohoto měniče je vidět na obrázku \ref{fig_StepDown-schematic}.

% \begin{figure}[h]
%     \centering
%     \includegraphics{obrazky/stepdown_schematic.pdf}
%     \caption{Schéma zapojení step-down měniče.}
%     \label{fig_StepDown-schematic}
% \end{figure}

\subsection{Výběr lineárního stabilizátoru pro 3,3~V}

Jak již bylo zmíněno v~předchozích kapitolách, pro napájení AD převodníku bude třeba vybrat lineární stabilizátor s~malým úbytkem napětí, tzv. LDO. Bohužel nikde neexistuje přesná definice, co je LDO stabilizátor a~co už ne, je třeba toto označení považovat za ne zcela objektivní. V~praxi se tak obvykle označují stabilizátory s~úbytkem napětí zhruba do \SI{2}{\volt}.

Jako vhodný pro toto zapojení byl vybrán stabilizátor od výrobce Advanced Monolithic Systems AMS1117 \cite{dat_AMS1117} ve variantě s~pevně nastaveným výstupním napětím na \SI{3,3}{\volt}. Stabilizátor jako takový nepotřebuje pro svou funkci kromě filtračních kondenzátorů na vstupu a~výstupu žádné další externí součástky. Podle katalogového listu výrobce zaručuje rozptyl výstupního napětí maximálně $\pm\SI{10}{\milli\volt}$ a~potlačení vstupního rušení o~alespoň \SI{60}{\deci\bel}, což je pro AD převodník žádaná vlastnost. Filtrační kondenzátory byly zvoleny podle doporučení výrobce \SI{22}{\micro\farad} na vstupu i~na výstupu v~keramickém provedení. Kvůli úspoře energie bude možnost vypínat tento stabilizátor a~tím i~následně AD převodník. Výsledné schéma zapojení lineárního stabilizátoru je vidět na obrázku \ref{fig_LDO-schematic}.

\begin{figure}[h]
    \centering
    \includegraphics[scale=1.2]{obrazky/ldo-schematic.pdf}
    \caption{Schéma zapojení lineárního stabilizátoru s~výstupem \SI{3.3}{\volt}.}
    \label{fig_LDO-schematic}
\end{figure}

\subsection{Výběr lineárního stabilizátoru pro 1,8~V}

Posledním z~napájecích zdrojů je lineární stabilizátor pro senzor SGP30 měřící CO$_2$ a~TVOC. I~zde bude třeba vybrat LDO stabilizátor, ideálně v~malém pouzdře s~minimem externích součástek. Vstupní napětí bude \SI{3}{\volt} až \SI{3.6}{\volt} z~baterie a~výstupní napětí musí být \SI{1.8}{\volt}, stabilizátor musí být schopen dodat proud alespoň \SI{50}{\milli\ampere}. Jako nejvhodnější, dostupný a~zároveň levný byl vybrán stabilizátor NCP115 s~fixním výstupem \SI{1.8}{\volt} od firmy ON Semiconductor \cite{dat_NCP115}. Tento stabilizátor vyžaduje přidání pouze filtračních kondenzátorů na vstup a~výstup, ideálně keramických o~hodnotě \SI{1}{\micro\farad}. Vstupní napětí může být až \SI{5.5}{\volt} a~výstupní proud \SI{300}{\milli\ampere}. Výsledné zapojení tohoto stabilizátoru včetně možnosti vypnutí je vidět na obrázku \ref{fig_LDO1.8-schematic}.

\begin{figure}[h]
    \centering
    \includegraphics[scale=1.2]{obrazky/ldo18-schematic.pdf}
    \caption{Zapojení lineárního stabilizátoru s~výstupem \SI{1.8}{\volt}.}
    \label{fig_LDO1.8-schematic}
\end{figure}



\section{Výsledné zapojení zařízení}

Na obrázku \ref{fig_BlockDiagram-full} můžeme vidět finální blokové schéma celého zařízení podle kterého bude třeba nakreslit schéma a~následně navrhnout desku plošných spojů. Celé zařízení bude doplněno kromě zmíněných senzorů a~potřebných zdrojů také o~například ochranu proti přepólování napájení. Dále bude deska na vstupu obsahovat obvody pro možnost zapnutí a~vypnutí celé elektroniky přes tlačítka a~také možnost vypnout zařízení pomocí mikrokontroleru. Pro možnost snazšího ladění obvodu budou také doplněny vypínatelné LED pro kontrolu přítomnosti jednotlivých napájecích napětí a~jedna LED s~tlačítkem pro možnost libovolného použití ve výsledné aplikaci. 

Jelikož je jedním z~hlavních úkolů navrhnout zařízení s~co nejmenší spotřebou, bude při návrhu potřeba brát ohled i~na mnoho dalších věcí. Všechny senzory a~části desky budou navrženy tak, aby je bylo možné nezávisle na sobě vypnout a~snížit tak spotřebu v~době, kdy není potřeba cokoli měřit. Pokud budou všechny senzory vypnuty, tvoří spotřebu celého zařízení samotný mikrokontroler ESP32 a~pasivní prvky, jako například pull-up rezistory a~napěťové děliče. Mikrokontroler ESP32 má v~deep-sleep (hluboký spánek) režimu spotřebu dle datasheetu \cite{dat_ESP32-WROOM} pouhých \SI{5}{\micro\ampere} a~zde již nelze spotřebu snížit. Jediné, co lze tedy ovlivnit jsou pasivní prvky a~to tak, že nebudou zapojeny přímo na napájecí větev \SI{3.3}{\volt} ale na vypínatelnou část obvodu a~tak nebudou zvyšovat klidovou spotřebu zařízení. Tímto způsobem budou tedy zapojeny všechny pull-up rezistory a~také napěťový dělič pro měření napětí na baterii.

\begin{figure}[h]
    \centering
    \includegraphics[width=\textwidth]{obrazky/block_schematic.png}
    \caption{Celkové blokové schéma navrhovaného zařízení.}
    \label{fig_BlockDiagram-full}
\end{figure}

\section{Nabíjení zařízení}

Jelikož je ve výsledném zařízení obsažena baterie LiFePO4, je vhodné mít k~dispozici kompatibilní nabíječku, ideálně připojitelnou k~tomuto zařízení, aby nebylo nutné vydělávat baterii. Pro tento účel se velmi často využívají moduly postavené na čipu TP5000 \cite{dat_TP5000}. Tento nabíjecí obvod má uvnitř spínaný zdroj a~umožňuje tedy nabíjení proudem až \SI{2}{\ampere} při vstupním napětí až \SI{10}{\volt}. Je tedy možné na vstup připojit i~solární panel o~vhodném výkonu a~parametrech a~nabíjet baterii z~energie jím vyrobené. Toto řešení má ovšem jednu nevýhodu a~to absenci MPPT (Maximum Power Point Tracking) obvodu pro maximalizaci získané energie ze solárního panelu. Kvůli absenci tohoto obvodu jsem se rozhodl pro realizaci vlastního nabíjecího modulu.

Vybral jsem řídící čip LT3652 od firmy Analog Devices \cite{dat_LT3652}, který sdružuje funkce nabíjení baterie a~MPPT regulátoru na vstupu. Samotný čip umožňuje nabíjení baterie spínaným zdrojem s~výstupním proudem až \SI{2}{\ampere} při vstupním napětí až \SI{32}{\volt}. Je možné tedy nabíjet i~více baterií zapojených do série. Umožňuje nastavit maximální napětí, do kterého se baterie mají nabíjet, nabíjecí proud a~nebo také právě napětí pro MPPT regulátor.

Při návrhu zapojení nabíjecího obvodu s~čipem LT3652 budu vycházet z~datasheetu výrobce \cite{dat_LT3652} a~v~něm uvedeného typického zapojení. Výrobce v~tomto datasheetu uvádí přímo konkrétní zapojení pro jednu LiFePO4 baterii s~maximálním napětím \SI{3.6}{\volt}, ze kterého budu vycházet. Toto zapojení je vidět na obrázku \ref{fig_datasheetSchematicLT3652}.

\begin{figure}[h]
    \centering
    \includegraphics[width=0.9\textwidth]{obrazky/datasheetSchematicLT3652.png}
    \caption{Zapojení nabíjecího obvodu LT3652 podle doporučení výrobce \cite{dat_LT3652}.}
    \label{fig_datasheetSchematicLT3652}
\end{figure}

Jako první bude doplněna možnost využití integrovaného MPPT regulátoru. Tento regulátor funguje tak, že se skrze napěťový dělič přivádí vstupní napětí na pin V$_{IN\_REG}$ obvodu, kde se porovnává s~referenčním napětím \SI{2.7}{\volt} a~pokud je toto napětí vyšší, umožňuje nabíjení maximálním proudem. Pokud je toto přivedené napětí nižší než \SI{2.7}{\volt}, dochází k~omezení nabíjecího proudu tak, aby byl stále zachován největší nabíjecí výkon. Poměr tohoto napěťového děliče lze spočítat dle následující rovnice:

\begin{equation}
    \frac{R_{IN1}}{R_{IN2}} = \frac{V_{IN\_min}}{2,7} - 1
    \label{eq_MPPTRegulator}
\end{equation}

Kde R$_{IN1}$ a~R$_{IN2}$ jsou odpory v~napěťovém děliči a~V$_{IN\_min}$ je minimální hodnota vstupního napětí pro nabíjení maximálním proudem.

Dalším doplněním zapojení od výrobce jsou signalizační LED. Obvod umí generovat na pinech CHRG a~FAULT až čtyři různé kombinace stavů. Na každém pinu tedy bude zapojena jedna LED, aby bylo možné tyto stavy rozeznat. Pokud nebude svítit ani jedna, neprobíhá žádné nabíjení a~nebo není připojena baterie, pokud svítí LED na pinu CHRG, tak probíhá normální proces nabíjení baterie. Pokud se rozsvítí LED na pinu FAULT, obvod vyhodnotil baterii jako špatnou. Tento stav nastává, pokud přetrvává i~přes zahájený nabíjecí cyklus podvybití baterie pod \SI{2.5}{\volt}. Posledním možným stavem je rozsvícení obou LED, které značí přehřívání baterie. V~této aplikaci nebude NTC (Negative Temperature Coefficient) senzor pro snímání teploty baterie zapojen a~tak nemůže tento stav nastat. Je zde doplněna ještě jedna indikační LED, která se rozsvítí ihned po přivedení napájecího napětí a~slouží tak ke kontrole zapnutí celého obvodu.

Poslední možností, jak doporučené zapojení upravit je změna nabíjecího proudu. Velikost nabíjecího proudu je dána velikostí odporu zapojeného v~sérii mezi kladným pólem baterie a~pinem SENSE obvodu. Velikost tohoto nastavovacího odporu lze vypočítat podle rovnice:

\begin{equation}
    R_{SENSE} = \frac{0,1}{I_{CHG(max)}}
    \label{eq_RsenseCharger}
\end{equation}

Kde I$_{CHG(max)}$ je právě velikost požadovaného nabíjecího proudu. Pro nabíjecí proud \SI{1}{\ampere} by vycházela hodnota odporu na \SI{100}{\milli\ohm}. Při vybírání konkrétních součástek je potřeba brát v~úvahu také výkonovou ztrátu na tomto odporu, jelikož skrze něj protéká veškerý nabíjecí proud, a~podle toho jej dostatečně výkonově dimenzovat. Celkové výsledné použité zapojení nabíječky je vidět na obrázku \ref{fig_ChargerSchematic}.

\begin{figure}
    \centering
    \includegraphics[width=\textwidth]{obrazky/batteryCharger_schematic.pdf}
    \caption{Výsledné zapojení nabíječky LiFePO4 baterie za použití obvodu LT3652.}
    \label{fig_ChargerSchematic}
\end{figure}


% \subsection{Anténa pro LoRa modul}

% Pro tuto práci jsem se rozhodl v~rámci potenciální minimalizace a~také snížení ceny výsledného produktu využít možnosti integrovat anténu pro LoRa modul přímo na desku plošných spojů. Na internetu existuje mnoho různých návrhů tzv. elektricky malých antén. Já jsem vybral od firmy Texas Instruments jejich návrh antény s~označením DN038 \cite{dat_DN038}. Jedná se o~elektricky malou anténu navrženou pro frekvenci \SI{868}{\mega\hertz} s~velice kompaktními rozměry $12\times\SI{19}{\milli\metre}$. Jedinou nevýhodou a~zároveň nutností je její přizpůsobení. Vysílací modul RFM95W má koncový stupeň o~impedanci \SI{50}{\ohm} a~tomu je třeba přizpůsobit i~danou anténu. Pro tyto účely je zapotřebí na desku plošných spojů umístit přizpůsobovací obvody v~podobě L nebo $\pi$ článku. Jelikož chci, aby bylo možné využít v~případě potřeby lepší externí antény, bude na desce možnost pomocí připájení \SI{100}{\pico\farad} kondenzátoru výběru mezi integrovanou anténou nebo připojením externí přes SMA konektor.