\chapter{Server pro zpracování naměřených dat}

Důležitým rozhodnutím pro realizaci celého zařízení je vhodný výběr aplikací, ve kterých se budou naměřená data uchovávat a následně zpracovávat či zobrazovat. Na trhu existuje několik veřejně dostupných serverů, které umožňují přijímání dat skrze různé protokoly a jejich následné uchovávání a zpracovávání.

Mezi nejznámější služby patří ThingSpeak\footnote{https://thingspeak.com/}, který umožňuje integraci MATLAB skriptů které se spouštějí nad uloženými přijatými daty. Při využívání neplacené verze této služby jsme limitování maximálním počtem přijatých zpráv za den (\SI{8200}{}), maximální dobou běhu skriptů \SI{20}{\second} nebo třeba pouze čtyřmi neveřejnými kanály na jeden účet. Další nevýhodou je možnost přijímat do jednoho kanálu maximálně 8 proměnných, pokud bychom chtěli vizualizovat více dat, musíme je rozdělit do více kanálů a můžeme tak brzy narazit na limity účtu poskytovaného zdarma.

Dalším z možných serverů na příjem a zpracování dat je ubidots\footnote{https://ubidots.com/}. I tento server poskytuje licenci zdarma, která je určena pro nekomerční použití studenty a kutily, kteří si chtějí platformu vyzkoušet. Nachází se zde omezení například v počtu maximálně tří připojených zařízení a uchovávání dat po dobu maximálně jednoho měsíce. Každé připojené zařízení může zasílat ke zpracování maximálně \SI{10}{} měřených veličin.

Jednou z možností je využít Arduino Cloud\footnote{https://docs.arduino.cc/cloud/iot-cloud} od stejnojmenné společnosti Arduino. Jejich cloud nabízí také plán pro používání bez poplatků, zde se dostáváme na mnohem větší restrikce než u dříve zmíněných služeb. Připojena mohou být pouze dvě zařízení, zařízení musí být naprogramováno v jejich prostředí, jelikož není k dispozici API (Application Programming Interface) pro připojení jiných zařízení. Největší nevýhodou je uchovávání naměřených dat pouze jeden den, což je pro statistiky či sledování nepoužitelné.

Další z mnoha možností, jak uchovávat a zpracovávt naměřená data je vytvoření vlastního prostředí pro tyto účely. Lze použít spojení databázové aplikace, která bude uchovávat data (např. InfluxDB\footnote{https://www.influxdata.com/}) ke které bude připojen Eclipse Mosquitto\footnote{https://mosquitto.org/}, což je tzv. MQTT broker, který je potřebný k přijímání dat zasílaných skrze protokol MQTT. Tento broker lze poté připojit přes Node-RED\footnote{https://nodered.org/}, což je programovací nástroj určený k jednoduchému propojení zařízení s dalšími službami, lze jej tedy použít pro zpracování přijatých zpráv a jejich následné uložení do databáze. Statistiky a přehledy lze vykreslovat pomocí služby Grafana\footnote{https://grafana.com/}.