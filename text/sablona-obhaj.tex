% Soubory musí být v kódování, které je nastaveno v příkazu \usepackage[...]{inputenc}

\documentclass[%        Základní nastavení
  %draft,    				  % Testovací překlad
  12pt,       				% Velikost základního písma je 12 bodů
	t,                  % obsah slajdů bude vždy začínat od shora (nebude vertikálně centrovaný)
	aspectratio=1610,   % poměr stran bude 16:10 (všechny projektory v učebnách na Technické 12 Brno),
	                    % další volby jsou 43, 149, 169, 54, 32.
	unicode,						% Záložky a informace budou v kódování unicode
]{beamer}				    	% Dokument třídy 'zpráva', vhodná pro sazbu závěrečných prací s kapitolami
%\usepackage{etex}

\usepackage[utf8]		  % Kódování zdrojových souborů je v UTF-8
	{inputenc}					% Balíček pro nastavení kódování zdrojových souborů
	
\usepackage{graphicx} % Balíček 'graphicx' pro vkládání obrázků
											% Nutné pro vložení logotypů školy a fakulty

\usepackage[          % Balíček 'acronym' pro sazby zkratek a symbolů
	nohyperlinks				% Nebudou tvořeny hypertextové odkazy do seznamu zkratek
]{acronym}						
											% Nutné pro použití prostředí 'acronym' balíčku 'thesis'

%% Balíček hyperref je volán třídou beamer automaticky, proto není třeba následujícího kódu:
%\usepackage[
%	breaklinks=true,		% Hypertextové odkazy mohou obsahovat zalomení řádku
%	hypertexnames=false % Názvy hypertextových odkazů budou tvořeny
%											% nezávisle na názvech TeXu
%]{hyperref}						% Balíček 'hyperref' pro sazbu hypertextových odkazů
%											% Nutné pro použití příkazu 'nastavenipdf' balíčku 'thesis'

\usepackage{cmap} 		% Balíček cmap zajišťuje, že PDF vytvořené `pdflatexem' je
											% plně "prohledávatelné" a "kopírovatelné"

%\usepackage{upgreek}	% Balíček pro sazbu stojatých řeckých písmem
											%% např. stojaté pí: \uppi
											%% např. stojaté mí: \upmu (použitelné třeba v mikrometrech)
											%% pozor, grafická nekompatibilita s fonty typu Computer Modern!

%\usepackage{amsmath} %balíček pro sabu náročnější matematiky

\usepackage{booktabs} % Balíček, který umožňuje v tabulce používat
                      % příkazy \toprule, \midrule, \bottomrule


%%%%%%%%%%%%%%%%%%%%%%%%%%%%%%%%%%%%%%%%%%%%%%%%%%%%%%%%%%%%%%%%%
%%%%%%      Definice informací o dokumentu             %%%%%%%%%%
%%%%%%%%%%%%%%%%%%%%%%%%%%%%%%%%%%%%%%%%%%%%%%%%%%%%%%%%%%%%%%%%%

% V tomto souboru se nastavují téměř veškeré informace, proměnné mezi studenty:
% jméno, název práce, pohlaví atd.
% Tento soubor je SDÍLENÝ mezi textem práce a prezentací k obhajobě -- netřeba něco nastavovat na dvou místech.

\usepackage[
%%% Z následujících voleb jazyka lze použít pouze jednu
  czech-english,		% originální jazyk je čeština, překlad je anglicky (výchozí)
  %english-czech,	% originální jazyk je angličtina, překlad je česky
  %slovak-english,	% originální jazyk je slovenština, překlad je anglicky
  %english-slovak,	% originální jazyk je angličtina, překlad je slovensky
%
%%% Z následujících voleb typu práce lze použít pouze jednu
  %semestral,		  % semestrální práce (nesází se abstrakty, prohlášení, poděkování) (výchozí)
  bachelor,			%	bakalářská práce
  %master,			  % diplomová práce
  %treatise,			% pojednání o disertační práci
  %doctoral,			% disertační práce
%
%%% Z následujících voleb zarovnání objektů lze použít pouze jednu
%  left,				  % rovnice a popisky plovoucích objektů budou zarovnány vlevo
	center,			    % rovnice a popisky plovoucích objektů budou zarovnány na střed (vychozi)
%
]{thesis}   % Balíček pro sazbu studentských prací


%%% Jméno a příjmení autora ve tvaru
%  [tituly před jménem]{Křestní}{Příjmení}[tituly za jménem]
% Pokud osoba nemá titul před/za jménem, smažte celý řetězec '[...]'
\author{Martin}{Kousal}

%%% Identifikační číslo autora (VUT ID)
\butid{221063}

%%% Pohlaví autora/autorky
% (nepoužije se ve variantě english-czech ani english-slovak)
% Číselná hodnota: 1...žena, 0...muž
\gender{0}

%%% Jméno a příjmení vedoucího/školitele včetně titulů
%  [tituly před jménem]{Křestní}{Příjmení}[tituly za jménem]
% Pokud osoba nemá titul před/za jménem, smažte celý řetězec '[...]'
\advisor[doc. Ing.]{Tomáš}{Frýza}[Ph.D.]

%%% Jméno a příjmení oponenta včetně titulů
%  [tituly před jménem]{Křestní}{Příjmení}[tituly za jménem]
% Pokud osoba nemá titul před/za jménem, smažte celý řetězec '[...]'
% Nastavení oponenta se uplatní pouze v prezentaci k obhajobě;
% v případě, že nechcete, aby se na titulním snímku prezentace zobrazoval oponent, pouze příkaz zakomentujte;
% u obhajoby semestrální práce se oponent nezobrazuje (jelikož neexistuje)
% U dizertační práce jsou typicky dva až tři oponenti. Pokud je chcete mít na titulním slajdu, prosím ručně odkomentujte a upravte jejich jména v definici "VUT title page" v souboru thesis.sty.
\opponent[doc.\ Mgr.]{Křestní}{Příjmení}[Ph.D.]

%%% Název práce
%  Parametr ve složených závorkách {} je název v originálním jazyce,
%  parametr v hranatých závorkách [] je překlad (podle toho jaký je originální jazyk).
%  V případě, že název Vaší práce je dlouhý a nevleze se celý do zápatí prezentace, použijte příkaz
%  \def\insertshorttitle{Zkác.\ náz.\ práce}
%  kde jako parametr vyplníte zkrácený název. Pokud nechcete zkracovat název, budete muset předefinovat,
%  jak se vytváří patička slidu. Viz odkaz: https://bit.ly/3EJTp5A
\title[IoT air monitoring]{IoT monitoring ovzduší}

%%% Označení oboru studia
%  Parametr ve složených závorkách {} je název oboru v originálním jazyce,
%  parametr v hranatých závorkách [] je překlad
\specialization[Electronics and Communication Technologies]{Elektronika a komunikační technologie}

%%% Označení ústavu
%  Parametr ve složených závorkách {} je název ústavu v originálním jazyce,
%  parametr v hranatých závorkách [] je překlad
%\department[Department of Control and Instrumentation]{Ústav automatizace a měřicí techniky}
%\department[Department of Biomedical Engineering]{Ústav biomedicínského inženýrství}
%\department[Department of Electrical Power Engineering]{Ústav elektroenergetiky}
%\department[Department of Electrical and Electronic Technology]{Ústav elektrotechnologie}
%\department[Department of Physics]{Ústav fyziky}
%\department[Department of Foreign Languages]{Ústav jazyků}
%\department[Department of Mathematics]{Ústav matematiky}
%\department[Department of Microelectronics]{Ústav mikroelektroniky}
\department[Department of Radio Electronics]{Ústav radioelektroniky}
%\department[Department of Theoretical and Experimental Electrical Engineering]{Ústav teoretické a experimentální elektrotechniky}
%\department[Department of Telecommunications]{Ústav telekomunikací}
%\department[Department of Power Electrical and Electronic Engineering]{Ústav výkonové elektrotechniky a elektroniky}

%%% Označení fakulty
%  Parametr ve složených závorkách {} je název fakulty v originálním jazyce,
%  parametr v hranatých závorkách [] je překlad
%\faculty[Faculty of Architecture]{Fakulta architektury}
\faculty[Faculty of Electrical Engineering and~Communication]{Fakulta elektrotechniky a~komunikačních technologií}
%\faculty[Faculty of Chemistry]{Fakulta chemická}
%\faculty[Faculty of Information Technology]{Fakulta informačních technologií}
%\faculty[Faculty of Business and Management]{Fakulta podnikatelská}
%\faculty[Faculty of Civil Engineering]{Fakulta stavební}
%\faculty[Faculty of Mechanical Engineering]{Fakulta strojního inženýrství}
%\faculty[Faculty of Fine Arts]{Fakulta výtvarných umění}
%
%Nastavení logotypu (v hranatych zavorkach zkracene logo, ve slozenych plne):
\facultylogo[logo/FEKT_zkratka_barevne_PANTONE_CZ]{logo/logo_fekt_urel_cze}

%%% Rok odevzdání práce
\graduateyear{2022}
%%% Akademický rok odevzdání práce
\academicyear{2021/22}

%%% Datum obhajoby (uplatní se pouze v prezentaci k obhajobě)
\date{7.\,1.\,2022}

%%% Místo obhajoby
% Na titulních stránkách bude automaticky vysázeno VELKÝMI písmeny (pokud tyto stránky sází šablona)
\city{Brno}

%%% Abstrakt
\abstract[%
The purpose of this semestral work is to make a device that can measure air quality parameters and send them wirelessly to the server, where that measured data are processed and then shown to the user. The aim is to create a device with the lowest possible power consumption for the possibility of battery operation.
]{%
Tato semestrální práce se zabývá vytvořením zařízení pro měření kvalitativních parametrů ovzduší a které je následně bezdrátově přenáší na server, kde se naměřená data zpracovávají a zobrazují uživateli. Cílem je vytvořit zařízení s co nejnižším odběrem pro možnost provozu na akumulátor.
}

%%% Klíčová slova
\keywrds[%
IoT, ESP32, sensors, dust particles sensor, air quality, LoRa, gas sensor, temperature, humidity, light intensity, UV light
]{%
IoT, ESP32, senzory, senzor prachových částic, kvalita vzduchu, LoRa, senzor plynů, teplota, vlhkost, intenzita osvětlení, UV záření
}

%%% Poděkování
\acknowledgement{%
Rád bych poděkoval vedoucímu bakalářské práce
panu doc.~Ing.~Tomáši Frýzovi, Ph.D.\ za odborné vedení,
konzultace, trpělivost a~podnětné návrhy k~práci.
}%      % v tomto souboru doplňte údaje o sobě, o názvu práce...
                       % (tento soubor je sdílený s textem práce)

%%%%%%%%%%%%%%%%%%%%%%%%%%%%%%%%%%%%%%%%%%%%%%%%%%%%%%%%%%%%%%%%%%%%%%%%

%%%%%%%%%%%%%%%%%%%%%%%%%%%%%%%%%%%%%%%%%%%%%%%%%%%%%%%%%%%%%%%%%%%%%%%%
%%%%%%     Nastavení polí ve Vlastnostech dokumentu PDF      %%%%%%%%%%%
%%%%%%%%%%%%%%%%%%%%%%%%%%%%%%%%%%%%%%%%%%%%%%%%%%%%%%%%%%%%%%%%%%%%%%%%
%% Při vloženém balíčku 'hyperref' lze použít příkaz '\pdfsettings'
\pdfsettings
%  Nastavení polí je možné provést také ručně příkazem:
%\hypersetup{
%  pdftitle={Název studentské práce},    	% Pole 'Document Title'
%  pdfauthor={Autor studenstké práce},   	% Pole 'Author'
%  pdfsubject={Typ práce}, 						  	% Pole 'Subject'
%  pdfkeywords={Klíčová slova}           	% Pole 'Keywords'
%}
\hypersetup{pdfpagemode=FullScreen}       % otevření rovnou v režimu celé obrazovky
%%%%%%%%%%%%%%%%%%%%%%%%%%%%%%%%%%%%%%%%%%%%%%%%%%%%%%%%%%%%%%%%%%%%%%%

\usetheme{VUT} 				% barvy a rozložení prezentace odpovídající VUT FEKT
% alternativně lze použít jiná berevná témata, ale bez záruky. Například: 
%\usetheme{Darmstadt} \usecolortheme{default2}
\logoheader					% vytvoření zkráceného loga VUT FEKT v hlavičce slajdu, nechte odkomentované



\begin{document}

% v případě zakomentování následujícího se zobrazí v pravém dolním rohu slajdů klikatelné navigační symboly 
\disablenavigationsymbols

% titulní snímek, vysazen bez horních, dolních a postranních lišt (volba plain),
% není tak vysazen ani nadpis snímku
\maketitle

%%%%%%%%%%%%%%%%%%%%%%%%%%%%%%%%%%%%%%%%%%%%%%%%%%%%%%%%%%%%%%%%%%%%%%%
% 1. snímek s cíli (zadaním) práce
\begin{frame} 
	% nadpis snímku
	\frametitle{Cíle práce}
	\begin{columns}[T]
		\begin{column}{0.4\textwidth}
			\vspace{0.5cm}
			\begin{itemize}
				\item Prostudovat dostupné senzory
				\item Vybrat vhodné senzory
				\item Prostudovat možnosti přenosu dat
				\item Navrhnout obvodové zapojení
				\item Realizovat desku plošných spojů
				\item Oživit a naprogramovat zařízení
    			\item Vizualizovat naměřená data
			\end{itemize}
		\end{column}

		\begin{column}{0.6\textwidth}
			\centering
			\vspace{1.5cm}
			\includegraphics[width=\columnwidth]{obrazky/block_schematic-blank.drawio.pdf}
		\end{column}
	\end{columns}
	
\end{frame}

%%%%%%%%%%%%%%%%%%%%%%%%%%%%%%%%%%%%%%%%%%%%%%%%%%%%%%%%%%%%%%%%%%%%%%%%%

\begin{frame}
	\frametitle{Návrh zařízení}
	\begin{columns}[T]
		\begin{column}{0.4\textwidth}
			Měřené veličiny
			\begin{itemize}
				\item Koncentrace CO$_{2}$
				\item Hodnota Total VOC
				\item Koncentrace prachových částic
				\item Intenzita osvětlení
				\item UV záření
			\end{itemize}
			Doplňkové veličiny
			\begin{itemize}
				\item Teplota
				\item Relativní vlhkost vzduchu
				\item Atmosférický tlak
			\end{itemize}	
		\end{column}

		\begin{column}{0.6\textwidth}
			\begin{figure}
				\centering
				\vspace{4ex}
				\includegraphics[width=\textwidth]{obrazky/block_schematic.png}
			\end{figure}
			
		\end{column}
	\end{columns}
\end{frame}

%%%%%%%%%%%%%%%%%%%%%%%%%%%%%%%%%%%%%%%%%%%%%%%%%%%%%%%%%%%%%%%%%%%%%%%%%

\begin{frame}
	\frametitle{Přenos a zpracování dat}
	Uvažované přenosové sítě
	\begin{itemize}
		\item Sigfox
		\item WiFi
		\item LoRaWAN
		\item NB-IoT
	\end{itemize}

	Zpracování dat
	\begin{itemize}
		\item Možnost využití platforem - ThingSpeak, ubidots, Arduino Cloud, \dots
		\item Zvoleno vlastní řešení - \textbf{databáze \texttt{+} Grafana}
	\end{itemize}

	\vspace{0.5ex}
	\begin{columns}
		\begin{column}{0.45\textwidth}
			LoRaWAN\\[0.5ex]
			\begin{itemize}
				\item The Things Network
				\item Node-RED
				\item InfluxDB
			\end{itemize}

		\end{column}

		\begin{column}{0.45\textwidth}
			WiFi\\[0.5ex]
			\begin{itemize}
				\item MQTT broker (Mosquitto)
				\item Node-RED
				\item InfluxDB
			\end{itemize}
		\end{column}
	\end{columns}
\end{frame}

%%%%%%%%%%%%%%%%%%%%%%%%%%%%%%%%%%%%%%%%%%%%%%%%%%%%%%%%%%%%%%%%%%%%%%%%%

\begin{frame}
	\frametitle{Napájení}

	\vspace{1ex}
	Při návrhu kladen důraz na:
	\begin{itemize}
		\item Možnost bateriového napájení
		\item Minimalizace počtu měničů napětí
		\item Co nejnižší spotřeba
	\end{itemize}

	\vspace{1.5ex}
	Výsledkem je:
	\begin{itemize}
		\item LiFePO4 baterie
		\item Nabíječka s možností MPPT
		\item Spotřeba v deep-sleep režimu $5,8~\mu A$
		\begin{itemize}
			\item Spotřeba ESP32 dle výrobce $5~\mu A$
			\item Vypínatelné senzory a pull-up rezistory
			\item ESP32 napájeno přímo z baterie bez měniče
		\end{itemize}
	\end{itemize}


\end{frame}

%%%%%%%%%%%%%%%%%%%%%%%%%%%%%%%%%%%%%%%%%%%%%%%%%%%%%%%%%%%%%%%%%%%%%%%%%

% \begin{frame}
% 	\frametitle{Návrh DPS}
% 	\begin{itemize}
% 		\item Výsledkem jsou dvouvrstvé DPS
% 	\end{itemize}
% 	\begin{columns}
% 		\begin{column}{0.5\textwidth}
% 			\centering
% 			Hlavní deska $50x100~mm$
% 			\begin{figure}
% 				\centering
% 				\includegraphics[height=0.65\textheight]{obrazky/mainBoard-kicad.png}
% 			\end{figure}
% 		\end{column}

% 		\begin{column}{0.5\textwidth}
% 			\centering
% 			Nabíječka $40x40~mm$
% 			\begin{figure}
% 				\centering
% 				\vspace{3ex}
% 				\includegraphics[width=0.7\textwidth]{obrazky/batteryCharger-top_.png}
% 			\end{figure}

% 		\end{column}
% 	\end{columns}
% \end{frame}

%%%%%%%%%%%%%%%%%%%%%%%%%%%%%%%%%%%%%%%%%%%%%%%%%%%%%%%%%%%%%%%%%%%%%%%%%

\begin{frame}
	\frametitle{Realizovaná DPS}
	\begin{columns}
		\begin{column}{0.6\textwidth}
			\centering
			Hlavní deska $50x100~mm$
			\begin{figure}
				\centering
				\begin{figure}
					\includegraphics[width=0.85\textwidth]{obrazky/final_pcb_both.png}
				\end{figure}
			\end{figure}
		\end{column}

		\begin{column}{0.4\textwidth}
			\centering
			Nabíječka $40x40~mm$
			\begin{figure}
				\centering
				\vspace{8ex}
				\includegraphics[width=0.9\textwidth]{obrazky/batteryCharger-final.png}
			\end{figure}
		\end{column}
	\end{columns}
\end{frame}


%%%%%%%%%%%%%%%%%%%%%%%%%%%%%%%%%%%%%%%%%%%%%%%%%%%%%%%%%%%%%%%%%%%%%%%%%

\begin{frame}
	\frametitle{Finální zařízení}
	\centering
	\includegraphics[angle=90,origin=c,width=0.95\textwidth]{obrazky/finalDevice.jpg}	

\end{frame}

%%%%%%%%%%%%%%%%%%%%%%%%%%%%%%%%%%%%%%%%%%%%%%%%%%%%%%%%%%%%%%%%%%%%%%%%%

\begin{frame}
	\frametitle{Naměřená data}
	\begin{figure}
		\centering
		\includegraphics[width=\textwidth]{obrazky/grafana.png}
	\end{figure}


\end{frame}

%%%%%%%%%%%%%
% \begin{frame} 
% 	\frametitle{Klíčové nástroje}

% 	% prostředí 'alertblock', které slouží pro zdůraznění informace
% 	\begin{alertblock}{Pro práci je klíčový Eulerův vzorec}
% 		$$\eul^{\jmag x}=\cos x + \jmag\sin x$$
% 	\end{alertblock}

% 	\vspace{4ex}
% 	Eulerova identita je speciálním případem tohoto vzorce, jestliže dosadíme $x=\uppi$\,:

% 	% prostředí 'block', které slouží jako informativní
% 	\begin{block}{Eulerova identita}
% 		$$\eul^{\jmag \uppi}=\cos \uppi + \jmag\sin \uppi,$$\\
% 		odkud vyplývá
% 		$$\eul^{\jmag \uppi}+1=0.$$
% 	\end{block}
% \end{frame} 


%%%%%%%%%%%%%
% \begin{frame} 
% 	\frametitle{Plošný spoj}
	
% 	\begin{columns}[T] 								% prostředí sloupce s umístěním nahoře
% 		\begin{column}{0.4\textwidth}		% první sloupec
% 			Obrázek znázorňuje model:\\[2ex]
% 			%
% 			\begin{itemize}
% 				\item Deska
% 				\item Součástky
% 				\item Signály
% 				\item Napájení
% 			\end{itemize}
% 		\end{column}
% 		%
% 		\begin{column}{0.6\textwidth}		% druhý sloupec
% 			\begin{figure}%	
% 				\centering
% 				\vspace{1cm}	              % horizontální mezera
% 				\includegraphics[width=0.8\columnwidth]{obrazky/soucastky}
% 				%lze vložit popisek, ale povetšinou je to v prezentaci zbytečné
% 				%\caption{Popisek obrázku}%
% 				%\label{obr:ukazka}
% 			\end{figure}
% 		\end{column}
% 	\end{columns}											% ukončení prostředí sloupce
% \end{frame}


%%%%%%%%%%%%%
% \begin{frame} 
% 	\frametitle{Výsledky}
% 	\vspace{1cm}
% 	\begin{table}[]
% 		\centering
% 		\caption{Výsledky měření mobilních sítí}
% 		\label{tab:tabulka}
% 			\begin{tabular}{lcc}
% 				\toprule
% 					Technologie  & Rychlost stahování [kB/s] & Rychlost nahrávání [kB/s] \\
% 				\midrule
% 					GPRS (2,5G)	& 7,2 	& 3,6\\
% 					UMTS 3G     & 48 		& 48\\
% 					HSPA (3,5G)	&	1\,706	&	720\\
% 					LTE (4G) 		& 40\,750 & 10\,750\\
% 				\bottomrule                                       
% 			\end{tabular}
% 	\end{table}
% \end{frame}


%%%%%%%%%%%%%
\begin{frame} 
	\frametitle{Závěr}
	Splněné cíle
	\begin{itemize}
		\item Zařízení umožňuje měřit kvalitu ovzduší
  		\item Uživatel si může zobrazit naměřená data a statistiky
		\item Výsledná cena je okolo 1600 Kč
		\item Spotřeba zařízení umožňuje provoz na jednu 3200~mAh baterii s periodou měření 1~hodina po dobu až 160~dní
		\item Při zapojení solárního panelu je možný neomezeně dlouhý provoz
	\end{itemize}
\end{frame}


% podekovani
\begin{frame}[c] 
% bez nadpisu snímku
	\frametitle{\mbox{ }}
	\begin{center}
		{\Huge Děkuji za pozornost!}
	\end{center}
\end{frame}

% otázky oponenta
\frame{
\frametitle{Otázky oponenta}
	\emph{Jakým způsobem je měřen proudový odběr zařízení v režimu spánku a během aktivity, zmiňovaný v závěru? Uveďte výpočty teoretické životnosti baterie pro WiFi a pro LoRaWAN komunikaci.}\\[2ex]
	%
	Průměrná spotřeba při použití LoRaWAN:
	\begin{equation}
		I_{avg}= \frac{I_{wake}\cdot t_{wake} + I_{sleep}\cdot t_{sleep}}{t_{wake}+ t_{sleep}} = \frac{0,1 \cdot 30 + 5,8\cdot 10^{-6}\cdot 3600}{30 + 3600} = 832\: \mu A
	\end{equation}
	Teoretická výdrž akumulátoru:
	\begin{equation}
		\frac{C}{I_{avg}} = \frac{3200 mAh}{832 \mu A} \div 24 = 160,3\: dnů
	\end{equation}
	Pro WiFi je $I_{avg} = 997,5\: \mu A$ a teoretická výdrž je tedy $133,7$ dnů
}

\frame{
\frametitle{Otázky oponenta}
	\emph{Popište funkci I2C level shifteru (Q7 a Q8 na str. 67).}\\[2ex]

	\centering
	\includegraphics[height=0.7\textheight]{obrazky/level_shifter.pdf}

}

\end{document}
